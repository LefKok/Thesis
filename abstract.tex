\begin{abstractgr}
Στον 21ο αιώνα, ο ψηφιακός κόσμος έχει ενσωματωθεί στην καθημερινότητά μας. Έξυπνες συσκευές θα είναι σύντομα σε θέση να αποφασίζουν μόνες τους πώς θα ενεργήσουν ώστε να διευκολύνουν την ζωή μας.
Οι δράσεις αυτές θα πρέπει να βασίζονται στην χρήση γνώσης που αποκτήθηκε  από παρόμοιες συσκευές που αντιμετώπισαν    παρόμοια προβλήματα. Ωστόσο, καθώς αυτές οι συσκευές γίνονται όλο και ποιο αυτόνομες,
πιθανές επιθέσεις μπορεί να οδηγήσουν σε προβλήματα όχι μόνο στον ψηφιακό κόσμο αλλά και στον πραγματικό

Ο σκοπός της παρούσας διπλωματικής εργασίας είναι, ο σχεδιασμός και η υλοποίηση ενός decentralized συστήματος για την διαχείριση της εμπιστοσύνης των συσκευών μεταξύ τους στο πλαίσιο του
Κοινωνικού Διαδικτύου των Πραγμάτων (ΚΔτΠ)
 όπως αυτό απεικονίζεται από το project Cosmos .
 
Οι μέθοδοι διαχείρισης εμπιστοσύνης βασιζόμενη στη φήμη(reputation-based) έχουν χρησιμοποιηθεί με επιτυχία την προηγούμενη
δεκαετία κυρίως σε Peer-to-Peer συστήματα. Για το λόγο αυτό, η εργασία αυτή χτίστηκε επάνω σε 
state-of-the-art αλγορίθμους που χρησιμοποιούνται σε τέτοια συστήματα, με στόχο την περαιτέρω ανάπτυξη και τροποποίηση των
ιδεών ώστε να μπορούν να ενσωματωθούν στο πλαίσιο του Διαδικτύου των Πραγμάτων.
 
Αυτή η μελέτη προσδιορίζει τις πιθανές απειλές που μπορούν να προκύψουν στο ΚΔτΠ. Στην συνέχεια αναλύεται η προτεινόμενη αρχιτεκτονική του συστήματος διαχείρισης εμπιστοσύνης, η οποία είναι χτισμένη σε δύο παρατηρήσεις πάνω στης κοινωνικές αλληλεπιδράσεις των ανθρώπων.
 Πρώτον, όταν κάποιος θέλει να έχει πρόσβαση σε μια νέα υπηρεσία  ζητά προτάσεις από
τους φίλους του. Αυτή η διαδικασία χτίζει την φήμη (reputation) κάποιου. Δεύτερον, όταν κάποιος έχει αρκετές
αλληλεπιδράσεις με κάποιον άλλο, η φήμη δεν έχει σημασία πια. Εκεί έχει χτιστεί εμπιστοσύνη.
 
Μετά από κατάλληλη μοντελοποίηση αυτών των εννοιών χρησιμοποιώντας ιδέες από την Θεωρία των Πιθανοτήτων, προσθέσουμε χαρακτηριστικά, όπως
η ικανότητα μίας κακόβουλης Εικονική Οντότητα (ΕΟ) να εξιλεωθεί , η γρήγορη αναγνώριση 
αλλαγών στη συμπεριφορά των ΕΟ από αξιόπιστες σε κακόβουλες και η δυναμική ενσωμάτωση 
νέων ΕΟ στο σύστημα. Στο τέλος τρέξαμε μια προσομοίωση και παρουσιάζουμε τα αποτελέσματα.
	
    \begin{keywordsgr}
   εμπιστοσύνη, φήμη, ασφάλεια , Διαδίκτυο των πραγμάτων, Εικονική οντότητα, Cosmos, ασφάλεια κατανεμημένων συστημάτων, Κοινωνικό Διαδίκτυο των Πραγμάτων, κλιμάκωση
	\end{keywordsgr}
	
	
\end{abstractgr}

\begin{abstracten}
In the 21st century the digital world is incorporated on everyday life. Smart devices will soon be able to decide on their own of actions needed to be taken in order to facilitate our lives. 
These actions will be based on acquiring knowledge from similar devices that have encountered similar problems. However, as these devices are becoming autonomous, 
potential attacks can result in problems not only in the digital world but also in the physical one.  
 
The purpose of this thesis is the design and implementation of a decentralized system for Trust Management in the context of the ​
Social Internet of Things​
 as seen by the ​Cosmos​ project. 
 
Reputation-­based trust management methods have been successfully used in the  past 
decade mostly on Peer-­to-­Peer systems. For this reason, this study is built upon 
state-­of-­the-­art algorithms used on such systems, with the aim of further developing the 
ideas proposed and modifying them to fit the context of the Internet of Things. 
 
This study identifies the potential threats that can emerge in SIoT. After that the proposed 
architecture is analyzed. It is built upon two observations of the social interactions of humans.
 Firstly, when someone wants to access a new service he asks for referrals from 
his friends. This feedback is called Reputation. Secondly, when someone has enough 
interactions with someone else, the reputation does not matter any more. There has been built Trust. 
 
After appropriately modeling these notions using Probability Theory we add features like 
the ability of a malicious Virtual Entity (VE) to get redemption, the quick identification of 
behavioural changes of VE’s from trustworthy to malicious and the dynamic integration of 
new VE’s in the system. In the end we run a simulation and present the results.
	\begin{keywordsen}
    Internet of Things, trust, reputation, scalability, security, Cosmos, Social Internet of Things, Distributed Systems security 
	\end{keywordsen}
\end{abstracten}

\begin{acknowledgementsgr}
===================================================================================

TODO

=============================================================================
\end{acknowledgementsgr}
