\chapter{Υλοποίηση προσομοίωσης RT-IOT με χρήση του TRMSim-WSN}\label{ch:implementation}

\section{TRMSim-WSN}

\diagramscale{TRMSim-WSN περιβάλλον εργασίας}{full_panel.png}{0.25}

Το TRMSim-WSN %cite
 είναι ,στο βαθμό που γνωρίζουμε, η state-of-the-art πλατφόρμα προσομοίωσης συστημάτων εμπιστοσύνης-φήμης. Δημιουργήθηκε για να προσομοιώνει αλγορίθμους διαχείρισης φήμης και εμπιστοσύνης σε wireless sensor networks συστήματα, αλλά οι ίδιες αρχές εφαρμόζονται και σε συστήματα Διαδικτύου των Πραγμάτων.
 Αυτό συμβαίνει επειδή και στο Διαδίκτυο των Πραγμάτων υπάρχουν διεσπαρμένοι κόμβοι στο χώρο με την διαφορά πως οι σχέσεις φιλίας δεν είναι βάση φυσικής απόστασης αλλά τυχαίες. 
 
 Η προσομοίωση γίνεται για μία υπηρεσία, άλλα λόγω της αποσύνδεσης των υπηρεσιών μεταξύ τους σε ότι αφορά το σύστημα διαχείρισης φήμης-εμπιστοσύνης, τα αποτελέσματα θα είναι ίδια και για πολλές υπηρεσίες. Ουσιαστικά ένα σύστημα με πολλές υπηρεσίες μπορεί να αναπαρασταθεί ως το άθροισμα πολλών ανεξάρτητων προσομοιώσεων με ίδιους κόμβους και διαφορετική κατανομή συμπεριφορών (malicious,benevolent,client). Παρακάτω παρουσιάζονται τα βασικά χαρακτηριστικά του προσομοιωτή.
 \newpage
 
 
 
Όπως φαίνεται στην εικόνα \ref{fig:settings_panel.png} το σύστημα έχει ένα πάνελ για ρύθμιση των παραμέτρων. Οι παράμετροι είναι:
\begin{itemize}
\item \textbf{Num executions:} Πόσες φορές κάνει request ένας πελάτης σε ένα συγκεκριμένο δίκτυο.
\item \textbf{Num networks:} Σε πόσα διαφορετικά τυχαία δίκτυα θα τρέξει η προσομοιώσει ώστε να ελεγχτεί η συμπεριφορά του συστήματος. Σε κάθε δίκτυο γίνονται Num executions αιτήσης για την υπηρεσία από την κάθε Εικονική Οντότητα.
\item \textbf{Min Num Sensors:} O ελάχιστος αριθμός Εικονικών Οντοτήτων ανά δίκτυο προσομοίωσης.
\item \textbf{Μax Num Sensors:} O μέγιστος αριθμός Εικονικών Οντοτήτων ανά δίκτυο προσομοίωσης.

\item \textbf{\% Clients:} Το ποσοστό των Εικονικών Οντοτήτων που δεν παρέχουν  υπηρεσίες.

\item \textbf{\% Relay Servers:} Το ποσοστό των Εικονικών Οντοτήτων που δέν παρέχει την συγκεκριμένη υπηρεσία, για τον σκοπό των δικών μας προσομοιώσεων συμπεριφέρεται σαν Client και για αυτό το κρατάμε στο 0\%.

\item \textbf{\% Malicous Servers:} Το ποσοστό των Εικονικών Οντοτήτων που παρέχουν κακόβουλες υπηρεσίες.

\item \textbf{Radio Range:} Η απόσταση που φτάνει το σήμα. Δεν έχει νόημα στο ΙοΤ και το κρατάμε στο 0.

\item \textbf{Delay:} Χρόνος καθυστέρησης μεταξύ δυο διαδοχικών αιτήσεων. Το χρησιμοποιούμε για καλύτερη εποπτεία κατά την διάρκεια της εκτέλεσης.

\item \textbf{Collusion:} Εάν είναι ενεργοποιημένο υπάρχουν κακόβουλες συνεργασίες στο σύστημα

\item \textbf{Oscilating:} Εάν είναι ενεργοποιημένο ανά τακτά χρονικά διαστήματα κάποιοι servers αλλάζουν συμπεριφορά απο καλόβουλη σε κακόβουλη. Τα ποσοστά(malicous,relay,clients) παραμένουν όμως σταθερά.

\item \textbf{Dynamic:} Εάν είναι ενεργοποιημένο ανά τακτά χρονικά διαστήματα κάποιες Εικονικές Οντότητες απενεργοποιούνται για 0.5sec προσομοιώνοντας την δυναμική ενεργοποίηση και απενεργοποίηση των πραγμάτων/αισθητήρων
\end{itemize}
\newpage
\diagramscale{Πάνελ Ρυθμίσεων}{settings_panel.png}{0.5}


Όπως φαίνεται στην εικόνα \ref{fig:network_panel.png} το σύστημα έχει ένα πάνελ για την αναπαράσταση των Εικονικών Οντοτήτων και των σχέσεων μεταξύ τους. Επειδή οι σχέσεις είναι μονόδρομες υπάρχει βέλος κατεύθυνσης. Παρακάτω θα δούμε πώς αλλάζουν αυτές οι σχέσεις ώστε οι κακόβουλες οντότητες να μην έχουν σχεδόν κανένα βέλος πάνω τους ενώ οι καλόβουλες να έχουν πολλά.

\diagramscale{Πάνελ Δικτύου}{network_panel.png}{0.4}
\newpage
Όπως φαίνεται στις εικόνες \ref{fig:outcome_panel_one.png}, \ref{fig:outcome_panel_many.png} το σύστημα έχει ένα πάνελ για την αναπαράσταση του μέσου όρου της ικανοποίησης των Εικονικών Οντοτήτων. Όταν η προσομοίωση γίνεται για ένα δίκτυο (RUN T\& R Model κουμπί) τότε η πράσινη γραμμή δείχνει την ικανοποίηση για κάθε κύκλο αιτήσεων (π.χ εάν Num execution = 30 τότε θα έχει 30 σημεία) και η κόκκινη τον μέσο όρο. Αντίθετα εάν γίνεται προσομοίωση για πολλά δίκτυα (Run Simulations κουμπί) τότε η πράσινη γραμμή δείχνει την ικανοποίηση μετά από την προσομοίωση σε ένα συγκεκριμένο δίκτυο και η κόκκινη πάλι την μέση τιμή. Δηλαδή εάν Num networks =100 , Num execution = 30 η πράσινη γραμμή θα έχει 100 σημεία όπου κάθε σημείο θα είναι το current satisfaction μετά από 30 εκτελέσεις σε ένα τυχαίο δίκτυο. Όπως είνα λογικό σε ένα δίκτυο χωρίς oscillating και dynamic το πρώτο πάνελ θα έχει μία αύξουσα συνάρτηση αφού τα δίκτυα θα προσκολλώνται σε έμπιστους φίλους ενώ το δεύτερο μία τυχαία με απότομες μεταβάσεις αφού τα δίκτυα δημιουργούνται με τυχαία διάταξη και αρχικές συνθήκες.


\diagramscale{Πάνελ ικανοποίησης σε ένα δίκτυο}{outcome_panel_one.png}{0.5}
\diagramscale{Πάνελ ικανοποίησης σε πολλά δίκτυα}{outcome_panel_many.png}{0.5}

\newpage
\section{Υλοποίηση RT-IOT}

Η Υλοποίηση του RT-IOT έγινε ως επέκταση του συστήματος προσομοίωσης και χρησιμοποιεί της δομές που παρέχει έτοιμες  το πακέτο TemplateTRM 
και οι οποίες κάνουν τα σωστά extend ώστε να μπορεί να λειτουργήσει ένα νέο σύστημα Trust \& Reputation πάνω στο TRMSim-WSN.
 Οι βασικές λειτουργικότητες της πλατφόρμας καθώς και η αρχικοποίηση του δικτύου προστέθηκαν στην κλάση TemplateTRM\_Network 
 ενώ οι λειτουργικότητες των Εικονικών Οντοτήτων προστέθηκαν στην κλάση TemplateTRM\_Sensor. 
 Επίσης για λόγους απομόνωσης,ευκολότερης υλοποίηση και επαναχρησιμοποίησης κώδικα δημιουργήθηκαν βοηθητικές κλάσεις της οποίες και παρουσιάζουμε παρακάτω:

%\javacode{lala}{Followee_struct.java}

\subsection{ComparableFriend \& FriendComparator}
 Η κλάση ComparableFriend \ref{lst:ComparableFriend.java} μοντελοποιεί μόνο τα απαραίτητα στοιχεία ενός φίλου για να μπορέσει να συγκριθεί με τους άλλους. Δηλαδή έχει το id που αναφέρεται στην εικονική οντότητα και το value με το οποίο θα καταταγεί, το οποίο ανάλογα την εφαρμογή μπορεί να πάρει τιμή δείκτη εμπιστοσύνης ή δείκτη φήμης.

 \javacode{ComparableFriend}{ComparableFriend.java}
 
Η κλάση FriendComparator \ref{lst:FriendComparator.java} είναι απλά ένας custom-made Comparator που χειρίζεται των τρόπο κατάταξης των φιλών σε μία PriorityQueue
 
 \javacode{FriendComparator}{FriendComparator.java}
 
\newpage


\subsection{MyOutcome,SatisfactionInterval,MyTransaction}

Η κλάση MyOutcome \ref{lst:MyOutcome.java} είναι ένα instance της abstract κλάσης Outcome που χρησιμοποιείτε από τον TRMSim-WSN για να αναπαραστήσει το αποτέλεσμα μίας συναλλαγής μεταξύ δύο Οντοτήτων.
 \javacode{MyOutcome}{MyOutcome.java}
 
 
  Τα βασικότερα στοιχεία αυτής της κλάσης είναι το Satisfaction που στην περίπτωσή μας έχει οριστεί ως SatisfactionInterval  \ref{lst:SatisfactionInterval.java} , δηλαδή παίρνει τιμές από 0 έως 1, και η μέθοδος aggregate  που παίρνει ένα collection από Outcomes και βγάζει την ολική ικανοποίηση το οποίο και προβάλετε στο Πάνελ ικανοποίησης \ref{fig:outcome_panel_one.png}
 \javacode{SatisfactionInterval}{SatisfactionInterval.java}

Tέλος οι κλάση MyTransaction \ref{lst:MyTransaction.java} περιεχέι τα βασικά στοιχεία μίας συναλλαγής όπως το βάρος,η εξασθένηση και φυσικά το outcome που περιεχέι την ικανοποίηση. Δηλαδή ένα MyTransaction Object αναπαριστά μία εγγραφή στο log\_file μίας εικονικής Οντότητας(μία γραμμή στον πίνακα \ref{tab:log file}).

 \javacode{MyTransaction}{MyTransaction.java}

\subsection{Followee\_struct}
Αυτή η κλάση αντιπροσωπεύει έναν φίλο (ή αλλιώς followee στο CosmoS). Περιεχέι όλες τις πληροφορίες που χρειάζεται να ξέρει μία Εικονική Οντότητα για κάποια άλλη, την οποία έχει φίλη για κάποια υπηρεσία, όπως το log\_file, τον δείκτη εμπιστοσύνης, τον δείκτη φήμης κ.α. (βλ. σχήμα \ref{fig:Followee_struct.png})

 Διαισθητικά, κοιτώντας την ιεραρχία στο σχήμα \ref{fig:files.png},  είναι λογικό πώς θα υπάρχουν 4 Followee\_struct. Αυτό σημαίνει 
 ότι θα υπάρχει διαφορετικό για την Εικονική Οντότητα Β στην υπηρεσία Vehicle και διαφορετικό  για την ίδια Ε.Ο. όταν παρέχει γνώση. 
 Αυτό είναι αποτέλεσμα της αποσύνδεσης των υπηρεσιών μεταξύ τους.
  Έτσι είναι δυνατός ο αποκλεισμός της Εικονικής Οντότητας για παροχή γνώσης χωρίς τον αποκλεισμό της για παροχή υπηρεσιών σε οχήματα.

\diagramscale{Followee}{Followee_struct.png}{0.5}

Εκτός από δεδομένα εσωκλείεται και λειτουργικότητα στο Followee\_struct
. Περιέχει μεθόδους για την εφαρμογή εξασθένησης στο log\_file, για τον υπολογισμό του δείκτη εμπιστοσύνης και 
βασικότερο όλων, περιεχέι την μέθοδο που υλοποιεί την αίτηση για την υπηρεσία στην Εικονική-Οντότητα φίλο.
 Μία περίληψη των μεθόδων φαίνεται στον κώδικα \ref{lst:Followee_struct.java}

 \javacode{Followee\_struct}{Followee_struct.java}


\subsection{Application\_struct}

H τελευταία βοηθητική κλάση που θα δούμε είναι το Application Struct. Σε αυτήν την κλάση συγκεντρώνουμε όλες τις λειτουργικότητες που χρειάζεται το RT-IOT για να μπορέσει να διαχειριστεί του δείκτες φήμης και εμπιστοσύνης μίας συγκεκριμένης υπηρεσίας. 
Άρα εδώ περιέχονται όλοι οι φίλοι που παρέχουν αυτήν την υπηρεσία καθώς και το κατώφλι που χρειάζεται να περνάει ο δείκτης εμπιστοσύνης για να εμπιστευθεί η Εικονική Οντότητα τον φίλο της. 

Διαισθητικά, κοιτώντας την ιεραρχία στο σχήμα \ref{fig:files.png},  είναι λογικό πώς θα υπάρχουν 2 Αpplication\_struct ένα για κάθε υπηρεσία τα οποία περιέχουν από 2 Followee\_struct το καθένα.

Επίσης εδώ περιέχεται η λογική για να συγκεντρωθούν οι διάφοροι δείκτες φήμης και εμπιστοσύνης των φίλων και να καταταγούν ώστε να βρεθεί ο πιο έμπιστος τόσο για ιδία χρήση μετά από σύγκριση με το threshold όσο και για παροχή recommendation.
 Ακόμα εδώ γίνεται ή εισαγωγή νέων εγγραφών στην κορυφή των log\_files, όπως φαίνεται στο \ref{lst:Application_struct.java}

 \javacode{Application\_struct}{Application_struct.java}


\subsection{Η πλατφόρμα}

H πλατφόρμα έχει δύο βασικές λειτουργικότητες. Πρώτον αρχικοποιεί το σύστημα και δεύτερον παρέχει προτάσεις για νέου φίλους με βάση κάποιον δείκτη φήμης που υπολογίζει.

\subsubsection{Αρχικοποίηση συστήματος}

Κατά την αρχικοποίηση του συστήματος η Πλατφόρμα δημιουργεί ένα Application\_struct  για κάθε υπηρεσία όπως και για recommendation στα οποία κρατάει Followee\_struct για κάθε Εικονική Οντότητα που παρέχει την συγκεκριμένη υπηρεσία.
 Στην συνέχεια δέχεται ως είσοδο από το σύστημα τις παραμέτρους του πάνελ ρυθμίσεων \ref{fig:settings_panel.png} και δημιουργεί ένα δίκτυο με τα συγκεκριμένα χαρακτηριστικά. Με βάση τις προδιαγραφές του TRMSim-WSN θεωρείτε πώς οι malicious servers παρέχουν υπηρεσία ικανοποίησης 0 και οι benevolent server παρέχουν υπηρεσία ικανοποιησης 1.0.
  Στο τέλος προστίθενται σε κάθε Εικονική Οντότητα δύο τυχαίοι φίλοι για μπορέσει αν εκκινήσει σωστά.
\newpage
 \javacode{Αρχικοποίηση Πλατφόρμας}{Init.java}

\subsubsection{Παροχή Προτάσεων}

Αυτή η διαδικασία αποτελείτε από δύο διαφορετικά βήματα αναγκαία για την σωστή λειτουργία της πλατφόρμας.

\begin{enumerate}
\item \textbf{Αποδοχή feedback:} Αυτή η μέθοδος καλείτε περιοδικά από της Εικονικές Οντότητες και προσομοιώνει την αποστολή feedback στην πλατφόρμα. Αυτό γίνεται στέλνοντας την εμπιστοσύνη στις διάφορες Εικονικές Οντότητες φίλους για διαφορές υπηρεσίες. Η Πλατφόρμα με την σειρά της τα αποθηκεύει στα δικά της Application\_stuct για περαιτέρω επεξεργασία.

\item\textbf{Παροχή προτάσεων:} Όταν ζητηθεί από την πλατφόρμα να προτείνει κάποιον νέο φίλο αυτή υπολογίζει τους δείκτες φήμης όλων τον Εικονικών Οντοτήτων που παρέχουν την υπηρεσία και τους κατατάσσει όμοια με την διαδικασία υπολογισμού εμπιστοσύνης. Στην συνέχει όπως αναφέραμε υπάρχει 80\% πιθανότητα να προτείνει μία από της πιο φημισμένες Εικονικές Οντότητες και 20\% να προτείνει μία τυχαία.
\end{enumerate}

 \javacode{Παροχή Προτάσεων}{Rec_Rep.java}


\subsection{H Εικονική Οντότητα}

Τα μόνα πεδία που χρειάζεται μία Εικονική Οντότητα είναι ένα HashMap με όλα τα\\
 Application\_struct για τις υπηρεσίες που ζητάει και ένα HashMap που περιεχέι της υπηρεσίες τις οποίες παρέχει.
 Στην προσομοίωση οι υπηρεσίες είναι συνδεδεμένες με έναν αριθμό που είναι η ικανοποίηση που θα παρέχουν, αλλά στον πραγματικό κόσμο εκεί θα υπάρχει μία δομή με την λογική για την παροχή της εκάστοτε υπηρεσίας.

Σε ότι αφορά την λογική που περιέχεται στις μεθόδους της κλάσης, αυτή αφορά αρκετά κομμάτια της λειτουργίας της Εικονική Οντότητας. 
Βασικότερη όλων είναι η μέθοδος run η οποία περιεχέι την λογική με την οποία θα γίνει η αίτηση στον κατάλληλο πάροχο τον οποίο θα προσπαθήσει αρχικά η Εικονική Οντότητα να τον βρει μέσα από τις δικές της εμπειρίες χρησιμοποιώντας το Application\_struct. 
 Στην συνέχεια εάν αποτύχει, είτε θα ψάξει για assistance από φίλους, είτε θα ζητήσει να τις προταθούν φίλοι. 
 Αυτό θα γίνει κατανεμημένα μιλώντας με τις φιλικές Εικονικές Οντότητες που περιλαμβάνονται στο Application\_struct για την υπηρεσία recommendations.
 Εάν και αυτό αποτύχει τότε θα ερωτηθεί η πλατφόρμα για προτάσεις. 
 Πριν το τέλος κάθε κύκλου υπάρχει η πιθανότητα αποστολής report στην πλατφόρμα, η οποία ακολουθείτε από καθαρισμό της λίστας φίλων από κακόβουλες Εικονικές Οντότητες 
 
  \javacode{Μέθοδος run()}{run.java}

Επιπρόσθετα μέσα στην Εικονική Οντότητα είναι και η λογική για την απόκτηση προτάσεων για φίλους και για την εξαγωγή δεικτών φήμης μέσα από τον κοινωνικό της κύκλο ο οποίος τελικά οδηγεί στην αίτηση για υπηρεσία από τον πιο φημισμένο όπως περιγράφεται στην παράγραφο  \ref{sec:friend_acquisition}.
  \javacode{Μέθοδοι εύρεσης φίλων}{reputation_sens.java}


Τέλος μπορεί να βρεθεί και η λογική για την παροχή υπηρεσιών. Δηλαδή υπάρχουν και οι μέθοδοι που δίνουν απάντηση σε κάποια εικονική οντότητα όταν προσπαθεί να μαζέψει προτάσεις, όταν ψάχνει μία υπηρεσία μέσα από την μέθοδο αίτησης σε φίλο φίλου και φυσικά όταν ζητάει μία υπηρεσία οπότε κοιτάμε στο HashMap και εφόσον υπάρχει εγγραφή γυρίζεται η ικανοποίηση
  \javacode{Μέθοδοι παροχής υπηρεσιών}{serve_more.java}
\newpage
\section{Υλοποίηση χαρακτηριστικών προσομοιωτή}
Εκτός από την υλοποίηση του συστήματος στον προσομοιωτή πρέπει να υλοποιηθούν και άλλα χαρακτηριστικά που θα βοηθήσουν στην προσομοίωση των επιθέσεων . Αυτά είναι η εναλλαγή συμπεριφοράς (oscilattin VE's), η δυναμική είσοδος και έξοδος από το σύστημα (dynamic) και η συνεργασία μεταξύ κακοόβουλων οντοτήτων.

\subsubsection{Εναλλαγή συμπεριφοράς} 
Η εναλλαγή συμπεριφοράς υλοποιείτε κατευθείαν στην λογική του προσομοιωτή για ν είναι εύκολη η εναλλαγή στην αναπαράσταση τον κόμβων και να μπορεί να παραμείνει ίδιο το ποσοστό των κακόβουλων οντοτήτων σε όλη την προσομοίωση.

 Έτσι ανά 20 κύκλους κάποιες εικονικές Οντότητες αλλάζουν την ικανοποίηση στην παρεχόμενη υπηρεσία από καλή σε κακή και αντίστροφα. Με αυτόν τον τρόπο θα μπορέσουμε να παρατηρήσουμε την δυναμική αναδιοργάνωση τον σχέσεων φιλίας μεταξύ των κόμβων όπως θα δούμε στο επόμενο κεφάλαιο.
 
  \javacode{Εναλλαγή συμπεριφοράς}{oscillate.java}
   
   
 \subsubsection{Δυναμική Εκτέλεση}
 
 Για να υλοποιηθεί η δυναμική εκτέλεση έχουμε βάλει στα threads την ώρα που εξυπηρετούν μία αίτηση να ελέγχουν μία συνθήκη η οποία ισχύει κάθε 20 αιτήσεις. Όταν συμβεί αυτό τα threads και άρα οι Εικονικές Οντότητες που αναπαριστούν γίνονται ανενεργές. Έτσι για όσο χρόνο είναι ανενεργές οι Εικονικές Οντότητες δεν απαντάνε σε αιτήματα ακριβώς όπως θα συνέβαινε και στην πραγματικότητα. 
 
    \javacode{Δυναμική Εκτέλεση}{dynamic.java}


 \subsubsection{Κακόβουλες συνεργασίες}
 
 Οι κακόβουλες συνεργασίες επηρεάζουν διαφορές λειτουργίες του συστήματος και για αυτό έπρεπε να αλλαχτούν αρκετές μέθοδοι. Έτσι :
 
 \begin{enumerate}
 \item Οι κακόβουλες εικονικές οντότητες δεν διαγράφουν τους κακόβουλους φίλους τους επειδή τους χρησιμοποιούν για να χειραγωγήσουν το σύστημα
 \item Όταν ζητάει κάποιος recommendation από κακόβουλη Ε.Ο. αυτή γυρίζει την αντίθετη τιμή εμπιστοσύνης από την πραγματική. Έτσι δίνει κακές συστάσεις για καλές οντότητες με βάση την γνώση της και καλές στους κακόβουλους φίλους της.
 \item Όταν ζητάει κάποιος την υπηρεσία από φίλο φίλου η κακόβουλη Ε.Ο. γυρίζει την υπηρεσία την οποία παρέχει μία άλλη κακόβουλη Εικονική Οντότητα. Αυτό οδηγεί σε χαμηλή ικανοποίηση του αρχικού αιτούντα.
 \item Όταν στέλνει report στην πλατφόρμα η κακόβουλη Ε.Ο. όλοι οι δείκτες εμπιστοσύνης είναι αντίθετοι έτσι ώστε να δυσκολέψει την διαδικασία παροχής προτάσεων από την πλατφόρμα.
 \end{enumerate}


    \javacode{Κακόβουλες συνεργασίες}{collusion.java}
