\chapter{Υλοποίηση προσομοίωσης RT-IOT με χρήση του TRMSim-WSN}\label{ch:implementation}

\section{TRMSim-WSN}

\diagramscale{TRMSim-WSN περιβάλλον εργασίας}{full_panel.png}{0.25}

Το TRMSim-WSN %cite
 είναι ,στο βαθμό που γνωρίζουμε, η state-of-the-art πλατφόρμα προσομοίωσης συστημάτων εμπιστοσύνης-φήμης. Δημιουργήθηκε για να προσομοιώνει αλγορίθμους διαχείρισης φήμης και εμπιστοσύνης σε wireless sensor networks συστήματα, αλλά οι ίδιες αρχές εφαρμόζονται και σε συστήματα Διαδικτύου των Πραγμάτων.
 Αυτό συμβαίνει επειδή και στο Διαδίκτυο των Πραγμάτων υπάρχουν διεσπαρμένοι κόμβοι στο χώρο με την διαφορά πως οι σχέσεις φιλίας δεν είναι βάση φυσικής απόστασης αλλά τυχαίες. 
 
 Η προσομοίωση γίνεται για μία υπηρεσία, άλλα λόγω της αποσύνδεσης των υπηρεσιών μεταξύ τους σε ότι αφορά το σύστημα διαχείρισης φήμης-εμπιστοσύνης, τα αποτελέσματα θα είναι ίδια και για πολλές υπηρεσίες. Ουσιαστικά ένα σύστημα με πολλές υπηρεσίες μπορεί να αναπαρασταθεί ως το άθροισμα πολλών ανεξάρτητων προσομοιώσεων με ίδιους κόμβους και διαφορετική κατανομή συμπεριφορών (malicious,benevolent,client). Παρακάτω παρουσιάζονται τα βασικά χαρακτηριστικά του προσομοιωτή.
 \newpage
 
 
 
Όπως φαίνεται στην εικόνα \ref{fig:settings_panel.png} το σύστημα έχει ένα πάνελ για ρύθμιση των παραμέτρων. Οι παράμετροι είναι:
\begin{itemize}
\item \textbf{Num executions:} Πόσες φορές κάνει request ένας πελαης σε ένα συγκεκριμένο δίκτυο.
\item \textbf{Num networks:} Σε πόσα διαφορετικά τυχαία δίκτυα θα τρέξει η προσομοιώσει ώστε να ελεγχτεί η συμπεριφορά του συστήματος. Σε κάθε δίκτυο γίνονται Num executions αιτήσης για υπηρεσία από την κάθε Εικονική Οντότητα.
\item \textbf{Min Num Sensors:} O ελάχιστος αριθμός Εικονικών Οντοτήτων ανά δίκτυο προσομοίωσης.
\item \textbf{Μax Num Sensors:} O μέγιστος αριθμός Εικονικών Οντοτήτων ανά δίκτυο προσομοίωσης.

\item \textbf{\% Clients:} Το ποσοστό των Εικονικών Οντοτήτων που δεν παρέχουν  υπηρεσίες.

\item \textbf{\% Relay Servers:} Το ποσοστό των Εικονικών Οντοτήτων που δέν παρέχει την συγκεκριμένη υπηρεσία, για τον σκοπό των δικών μας προσομοιώσεων συμπεριφέρεται σαν Client και για αυτό το κρατάμε στο 0\%.

\item \textbf{\% Malicous Servers:} Το ποσοστό των Εικονικών Οντοτήτων που παρέχουν κακόβουλες υπηρεσίες.

\item \textbf{Radio Range:} Η απόσταση που φτάνει το σήμα. Δεν έχει νόημα στο ΙοΤ και το κρατάμε στο 0.

\item \textbf{Delay:} Χρόνος καθυστέρησης μεταξύ δυο διαδοχικών αιτήσεων. Το χρησιμοποιούμε για καλύτερη εποπτεία κατά την διάρκεια της εκτέλεσης.

\item \textbf{Collusion:} Εάν είναι ενεργοποιημένο υπάρχουν κακόβουλες συνεργασίες στο σύστημα

\item \textbf{Oscilating:} Εάν είναι ενεργοποιημένο ανά τακτά χρονικά διαστήματα κάποιοι servers αλλάζουν συμπεριφορά απο καλόβουλη σε κακόβουλη. Τα ποσοστά(malicous,relay,clients) παραμένουν όμως σταθερά.

\item \textbf{Dynamic:} Εάν είναι ενεργοποιημένο ανά τακτά χρονικά διαστήματα κάποιες Εικονικές Οντότητες απενεργοποιούνται για 0.5sec προσομοιώνοντας την δυναμική ενεργοποίηση και απενεργοποίηση των πραγμάτων/αισθητήρων
\end{itemize}
\newpage
\diagramscale{Πάνελ Ρυθμίσεων}{settings_panel.png}{0.5}


Όπως φαίνεται στην εικόνα \ref{fig:network_panel.png} το σύστημα έχει ένα πάνελ για την αναπαράσταση των Εικονικών Οντοτήτων και των σχέσεων μεταξύ τους. Επειδή οι σχέσεις είναι μονόδρομες υπάρχει βέλος κατεύθυνσης. Παρακάτω θα δούμε πώς αλλάζουν αυτές οι σχέσεις ώστε οι κακόβουλες οντότητες να μην έχουν σχεδόν κανένα βέλος πάνω τους ενώ οι καλόβουλες να έχουν πολλά.

\diagramscale{Πάνελ Δικτύου}{network_panel.png}{0.4}
\newpage
Όπως φαίνεται στις εικόνες \ref{fig:outcome_panel_one.png}, \ref{fig:outcome_panel_many.png} το σύστημα έχει ένα πάνελ για την αναπαράσταση του μέσου όρου της ικανοποίησης των Εικονικών Οντοτήτων. Όταν η προσομοίωση γίνεται για ένα δίκτυο (RUN T\& R Model κουμπί) τότε η πράσινη γραμμή δείχνει την ικανοποίηση για κάθε κύκλο αιτήσεων (π.χ εάν Num execution = 30 τότε θα έχει 30 σημεία) και η κόκκινη τον μέσο όρο. Αντίθετα εάν γίνεται προσομοίωση για πολλά δίκτυα (Run Simulations κουμπί) τότε η πράσινη γραμμή δείχνει την ικανοποίηση μετά από την προσομοίωση σε ένα συγκεκριμένο δίκτυο και η κόκκινη πάλι την μέση τιμή. Δηλαδή εάν Num networks =100 , Num execution = 30 η πράσινη γραμμή θα έχει 100 σημεία όπου κάθε σημείο θα είναι το current satisfaction μετά από 30 εκτελέσεις σε ένα τυχαίο δίκτυο. Όπως είνα λογικό σε ένα δίκτυο χωρίς oscillating και dynamic το πρώτο πάνελ θα έχει μία αύξουσα συνάρτηση αφού τα δίκτυα θα προσκολλώνται σε έμπιστους φίλους ενώ το δεύτερο μία τυχαία με απότομες μεταβάσεις αφού τα δίκτυα δημιουργούνται με τυχαία διάταξη και αρχικές συνθήκες.


\diagramscale{Πάνελ ικανοποίησης σε ένα δίκτυο}{outcome_panel_one.png}{0.5}
\diagramscale{Πάνελ ικανοποίησης σε πολλά δίκτυα}{outcome_panel_many.png}{0.5}

\newpage
\section{Υλοποίηση RT-IOT}

Η Υλοποίηση του RT-IOT έγινε ως επέκταση του συστήματος προσομοίωσης και χρησιμοποιεί της δομές του πακέτου TemplateTRM που παρέχει έτοιμες και οι οποίες κάνουν τα σωστά extend ώστε να μπορεί να λειτουργήσει ένα νέο σύστημα Trust \& Reputation πάνω στο TRMSim-WSN. Οι βασικές λειτουργικότητες της πλατφόρμας καθώς και η αρχικοποίηση του δικτύου προστέθηκαν στην κλάση TemplateTRM\_Network ενώ οι λειτουργικότητες των Εικονικών Οντοτήτων προστέθηκαν στην κλάση TemplateTRM\_Sensor. Επίσης για λόγους απομόνωσης,ευκολότερης υλοποίηση και επαναχρησιμοποίησης κώδικα δημιουργήθηκαν βοηθητικές κλάσεις της οποίες και παρουσιάζουμε παρακάτω:

%\javacode{lala}{Followee_struct.java}

\subsection{ComparableFriend \& FriendComparator}
 Η κλάση ComparableFriend \ref{lst:ComparableFriend.java} μοντελοποιεί μόνο τα απαραίτητα στοιχεία ενός φίλου για να μπορέσει να συγκριθεί με τους άλλους. Δηλαδή έχει το id που αναφέρεται στην εικονική οντότητα και το value με το οποίο θα καταταγεί, το οποίο ανάλογα την εφαρμογή μπορεί να πάρει τιμή δείκτη εμπιστοσύνης ή δείκτη φήμης.

 \javacode{ComparableFriend}{ComparableFriend.java}
 
Η κλάση FriendComparator \ref{lst:FriendComparator.java} είναι απλά ένας custom-made Comparator που χειρίζεται των τρόπο κατάταξης των φιλών σε μία PriorityQueue
 
 \javacode{FriendComparator}{FriendComparator.java}
 
\newpage


\subsection{MyOutcome,SatisfactionInterval,MyTransaction}

Η κλάση MyOutcome \ref{lst:MyOutcome.java} είναι ένα instance της abstract κλάσης Outcome που χρησιμοποιείτε από τον TRMSim-WSN για να αναπαραστήσει το αποτέλεσμα μίας συναλλαγής μεταξύ δύο Οντοτήτων.
 \javacode{MyOutcome}{MyOutcome.java}
 
 
  Τα βασικότερα στοιχεία αυτής της κλάσης είναι το Satisfaction που στην περίπτωσή μας έχει οριστεί ως SatisfactionInterval  \ref{lst:SatisfactionInterval.java} , δηλαδή παίρνει τιμές από 0 έως 1, και η μέθοδος aggregate  που παίρνει ένα collection από Outcomes και βγάζει την ολική ικανοποίηση το οποίο και προβάλετε στο Πάνελ ικανοποίησης \ref{fig:outcome_panel_one.png}
 \javacode{SatisfactionInterval}{SatisfactionInterval.java}

Tέλος οι κλάση MyTransaction \ref{lst:MyTransaction.java} περιεχέι τα βασικά στοιχεία μίας συναλλαγής όπως το βάρος,η εξασθένηση και φυσικά το outcome που περιεχέι την ικανοποίηση. Δηλαδή ένα MyTransaction Object αναπαριστά μία εγγραφή στο log\_file μίας εικονικής Οντότητας

 \javacode{MyTransaction}{MyTransaction.java}

\subsection{Followee\_struct}
Αυτή η κλάση αντιπροσωπεύει έναν φίλο (ή αλλιώς followee στο CosmoS). Περιεχέι όλες τις πληροφορίες που χρειάζεται να ξέρει μία Εικονική Οντότητα για κάποια άλλη, την οποία έχει φίλη για κάποια υπηρεσία, όπως το log\_file, τον δείκτη εμπιστοσύνης, τον δείκτη φήμης κ.α. (βλ. σχήμα \ref{fig:Followee_struct.png})

 Διαισθητικά, κοιτώντας την ιεραρχία στο σχήμα \ref{fig:files.png},  είναι λογικό πώς θα υπάρχουν 4 Followee\_struct. Αυτό σημαίνει 
 ότι θα υπάρχει διαφορετικό για την Εικονική Οντότητα Β στην υπηρεσία Vehicle και διαφορετικό  για την ίδια Ε.Ο. όταν παρέχει γνώση. 
 Αυτό είναι αποτέλεσμα της αποσύνδεσης των υπηρεσιών μεταξύ τους.
  Έτσι είναι δυνατός ο αποκλεισμός της Εικονικής Οντότητας για παροχή γνώσης χωρίς τον αποκλεισμό της για παροχή υπηρεσιών σε οχήματα.

\diagramscale{Followee}{Followee_struct.png}{0.5}

Εκτός από δεδομένα εσωκλείεται και λειτουργικότητα στο Followee\_struct
. Περιέχει μεθόδους για την εφαρμογή εξασθένησης στο log\_file, για τον υπολογισμό του δείκτη εμπιστοσύνης και 
βασικότερο όλων περιεχέι την μέθοδο που υλοποιεί την αίτηση για την υπηρεσία στην Εικονική-Οντότητα φίλο.
 Μία περίληψη των μεθόδων φένεται στον κώδικα \ref{lst:Followee_struct.java}

 \javacode{Followee\_struct}{Followee_struct.java}


\subsection{Application\_struct}

H Τελευταία βοηθητική κλάση που θα δούμε είναι το Application Struct. Σε αυτήν την κλάση συγκεντρώνουμε όλες τις λειτουργικότητες που χρειάζεται το RT-IOT για να μπορέσει να διαχειριστεί του δείκτες φήμης και εμπιστοσύνης μίας συγκεκριμένης υπηρεσίας. 
Άρα εδώ περιέχονται όλοι οι φίλοι που παρέχουν αυτήν την υπηρεσία καθώς και το κατώφλι που χρειάζεται να περνάει ο δείκτης εμπιστοσύνης για να εμπιστευθεί η Εικονική Οντότητα τον φίλο της. 

Διαισθητικά, κοιτώντας την ιεραρχία στο σχήμα \ref{fig:files.png},  είναι λογικό πώς θα υπάρχουν 2 Αpplication\_struct ένα για κάθε υπηρεσία τα οποία περιέχουν από 2 Followee\_struct το καθένα.

Επίσης εδώ περιέχεται η λογική για να συγκεντρωθούν οι διάφοροι δείκτες φήμης και εμπιστοσύνης των φίλων και να καταταγούν ώστε να βρεθεί ο πιο έμπιστος τόσο για ιδία χρήση μετά από σύγκριση με το threshold όσο και για παροχή recommendation.
 Ακόμα εδώ γίνεται ή εισαγωγή νέων εγγραφών στην κορυφή των log\_files, όπως φαίνεται στο \ref{lst:Application_struct.java}

 \javacode{Application\_struct}{Application_struct.java}


