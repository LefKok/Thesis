\chapter{Θεωρητικό και Τεχνολογικό Υπόβαθρο}\label{ch:bkg}
\section{Ορισμοί Εννοιών}

Τα συστήματα εμπιστοσύνης και φήμης αποτελούν μία εξέλιξη της τελευταίας δεκαπενταετίας και χρησιμοποιούνται ευρέως σε διαδικτυακές αλληλεπιδράσεις μεταξύ τόσο ανθρώπων (e-bay, amazon) όσων και μηχανών (ευφυών δραστών, peer-to-peer συστημάτων κτλ.). Η ιδέα πάνω στην οποία βασίζονται είναι: η συλλογή κριτικής για έναν δράστη από άλλα μέλη της κοινότητας τα οποία έχουν ήδη αποκτήσει κάποιες εμπειρίες με τον πρώτο. Με αυτόν τον τρόπο υπάρχει η δυνατότητα να γνωρίζει κάποιος για "το ποιόν"  κάποιου άλλου χωρίς να χρειάζεται να τον γνωρίσει, δηλαδή να αλληλεπιδράσει με αυτόν. Αυτή η δυνατότητα με την σειρά της "εξαναγκάζει" του δράστες ενός συστήμηματος να συμμορφωθούν με τους κανόνες της δικτυακή κοινότητας της οποίας είναι μέλη.

Όταν η αίτηση για παροχή υπηρεσιών ή για απόκτηση δεδομένων γίνεται online μεταξύ οντοτήτων πρότερα αγνώστων μεταξύ τους τότε μπορούν να προκύψουν 2 βασικά προβλήματα.
\begin{enumerate}
\item Το πρώτο πρόβλημα που δεν εξετάζεται στην παρούσα εργασία είναι πώς τα δεδομένα που παρέχονται (και στην περίπτωση του COSMOS η γνώση) μπορεί να χρησιμοποιηθούν από τον παραλήπτη για διαφορετικούς σκοπούς από αυτούς που ισχυρίζεται. Έτσι να τον βοηθήσουμε άθελα μας στου κακόβουλους ή γενικότερα αντίθετους από το καλό της κοινότητας σκοπούς του.

\item Αντίθετα αυτή η εργασία εστιάζει στο πρόβλημα του ρίσκου πού παίρνει ένας χρήστης όταν ζητάει μία υπηρεσία, επειδή μπορεί τελικά να μην την λάβει σε ικανοποιητικό βαθμό οπότε να πρέπει να την ξαναζητήσει. Ακόμα είναι δυνατόν να κινδυνέψει η ασφάλεια του (αντί για σωστό αρχείο να σταλεί κάτι που περιέχει ένα Trojan), ενώ στην δική μας περίπτωση μπορεί η γνώση και εν συνεχεία οι δράσεις που προτείνονται να προκαλέσουν προβλήματα ασφαλείας όχι μόνο στον ψηφιακό κόσμο  των Εικονικών Οντοτήτων, όσο και στο φυσικό κόσμο των πραγμάτων και των ανθρώπων.
\end{enumerate}
\newpage
\subsection{Soft Security}


Ως ασφάλεια στην πιο γενική της έννοια μπορούμε να ορίσουμε "την κατάσταση στην οποία κάποιος ή κάτι βρίσκεται μακριά από οποιονδήποτε κίνδυνο και οποιαδήποτε προσπάθεια αλλοίωσης αυτής της κατάστασης θα αποτύχει" (ή και να πετύχει θα είναι δυνατή η επαναφορά στην αρχική κατάσταση χωρίς κόστος). Αυτός ο ορισμός είναι πολύ γενικός και περιλαμβάνει την ασφάλεια ανθρώπων την ασφάλεια περιουσιών ακόμα και την ασφάλεια υγείας.

Στην επιστήμη των υπολογιστών η βασικότερη κατηγορία ασφάλειας είναι η ασφάλεια που παρέχει ως υπηρεσία τον αποκλεισμό δραστών εξωτερικών του συστήματος από την απόκτηση πρόσβασης σε προστατευόμενες πληροφορίες ή πόρους. Αυτό το είδος "hard security" παρέχεται μέσα από την πιστοποίηση και τον έλεγχο πρόσβασης(access control) στα υπολογιστικά συστήματα καθώς και μέσα από την κρυπτογραφία στα κανάλια επικοινωνίας κατά την ανταλλαγή εμπιστευτικών πληροφοριών. Όμως το πρόβλημα μπορεί να αντιστραφεί και τότε όλες αυτές οι μέθοδοι είναι αναξιόπιστες. Αυτό θα συμβεί στην περίπτωση που η επίθεση γίνει εντός του συστήματος. Μία τέτοια επίθεση συμβαίνει όταν κάποιος δράστης ή γενικότερα κομμάτι του συστήματος, που έχει δικαίωμα να ενεργεί στο σύστημα, γίνει κακόβουλο ή ελαττωματικό. 

Αυτά τα προβλήματα λύνονται μέσα από εφαρμογές του "Soft Security" το οποίο είναι 

\begin{soft_sec}
Η συνεργασία των δραστών ενός συστήματος ώστε να αποκλείσουν μη κοινωνικά αποδεχτές συμπεριφορές. 
\end{soft_sec}

Με αυτόν τον τρόπο μπορεί να προστατευθεί η κοινότητα από κακόβουλους δράστες οι οποίοι όμως έχουν κάθε δικαίωμα να ενεργούν. Ουσιαστικά να εφαρμοστεί ένα είδος κοινωνικού αποκλεισμού. Ένας από τους βασικότερους μηχανισμούς για την επίτευξή αυτού του είδους ασφαλείας είναι με την χρήση των εννοιών της εμπιστοσύνης και της φήμης. \footnote{Φυσικά και η έννοια της εμπιστοσύνης είναι χρήσιμη και για το "hard security" (εμπιστοσύνη σε ένα πιστοποιητικό ή μία Certificate Authority που το υπογράφει), αλλά ουσιαστικά αυτό είναι μία εφαρμογή "Soft Security" της ανθρώπινης κοινωνίας που χρησιμεύει στο hard security της υπολογιστικής επιστήμης.} 



\subsection{Η Έννοια της Εμπιστοσύνης}\label{sec:trust}

Στον κόσμο των ανθρώπινων οντοτήτων η ύπαρξη της έννοιας τις εμπιστοσύνης είναι εμφανής στις καθημερινές μας αλληλεπιδράσεις. Θα μιλήσουμε για τα προβλήματα μας και της εμπειρίες μας -δηλαδή θα δώσουμε ευαίσθητες προσωπικές πληροφορίες- στους φίλους μας οι οποίοι περιμένουμε να μας βοηθήσουνε με δικές τους εμπειρίες, καθώς και να βοηθηθούν από τις πληροφορίες μας. Επίσης θα πάμε να αγοράσουμε προϊόντα από μαγαζιά που γνωρίζουμε και έχουμε ξαναπάει επειδή πιστεύουμε πως η ποιότητα ου προϊόντος ή της υπηρεσίας που θα λάβουμε θα είναι ικανοποιητική για εμάς.

Στον κόσμο όμως της υπολογιστικής επιστήμης ο ορισμός και η μοντελοποίηση της εμπιστοσύνης είναι αρκετά δύσκολη. Αυτό συμβαίνει επειδή τις ανθρώπινες σχέσεις τις κυβερνούνε συναισθήματα όπως αγάπη και αφοσίωση που δεν έχουν καμία θέση στον ψηφιακό κόσμο. Στο έργο του Audun Jøsang %\cite survey trust josang
υπάρχει μία εξαιρετική συγκέντρωση για τους διαφόρους ορισμούς της εμπιστοσύνης και τις φήμης.
Εκεί λοιπόν αναφέρεται πώς υπάρχουν δύο βασικοί ορισμοί για την εμπιστοσύνη τους οποίους ονομάζουμε εμπιστοσύνη αξιοπιστίας (reliability trust) και εμπιστοσύνη απόφασης (decision trust)


Όπως προϊδεάζει και το όνομα η εμπιστοσύνη αξιοπιστίας μπορεί να ερμηνευθεί ως η αξιοπιστία κάποιου δράστη ή αντικειμένου, και ο ορισμός του Gambetta %\cite gambetta apo josang
 παρέχει ένα παράδειγμα του πώς μπορεί να διατυπωθεί:
\begin{reliability}
Εμπιστοσύνη είναι η υποκειμενική πιθανότητα από την οποία ένα άτομο, Α, περιμένει από ένα άλλο άτομο, Β, να κάνει μία συγκεκριμένη ενέργεια από την οποία εξαρτάται η ευημερία του.
\end{reliability}

Αυτός ορισμός περιεχέι την έννοια της εξάρτησης του Α από τον Β, και την αξιοπιστία (δηλαδή την πιθανότητα) του Β όπως την αντιλαμβάνεται ο Α.

Όμως η εμπιστοσύνη μπορεί να είναι πολύ πιο σύνθετη από τον παραπάνω ορισμό. Αυτό συμβαίνει επειδή το γεγονός ότι η αξιοπιστία ενός δράστη είναι υψηλή δεν οδηγεί στην τυφλή εμπιστοσύνη του για οποιαδήποτε ενέργεια. Το ρίσκο τις αποτυχίας μίας συγκεκριμένη συναλλαγής μπορεί να είναι πολύ μεγάλο. Δηλαδή με απλά λόγια, ένας δράστης μπορεί να θεωρηθεί αξιόπιστος για μία απλή συναλλαγή, όμως για κάποια πιο κρίσιμη η αξιοπιστία του να μην είναι αρκετά καθησυχαστική. Για να συμπεριλάβουμε λοιπόν την πιο ευρεία έννοια που μπορεί να χαρακτηριστεί ως εμπιστοσύνη, μπορεί να χρησιμοποιηθεί ο ακόλουθος ορισμός των  McKnight \& Chervany %\cite mcknight apo josang

\begin{decision}
Η εμπιστοσύνη είναι ο βαθμός στον οποίο ένα άτομο είναι πρόθυμο να εξαρτηθεί από κάτι ή κάποιον σε μια δεδομένη κατάσταση με ένα αίσθημα σχετικής ασφάλειας, ακόμη και αν οι αρνητικές συνέπειες είναι δυνατές.
\end{decision}

Ο γενικότερος αυτός ορισμός μας παρέχει μία χρήσιμη ασάφεια αφού περιλαμβάνει πτυχές μίας ευρύτερης έννοιας της εμπιστοσύνης, η οποία περιέχει της έννοιες της εξάρτησης από αυτόν που απολαμβάνει την εμπιστοσύνη, της αξιοπιστίας αυτού καθώς και της ύπαρξης ενός παράγοντα κινδύνου που πρέπει να αποδεχτεί το άτομο που καλείτε να εμπιστευτεί κάποιο άλλο άτομο, ανάλογο με την εκάστοτε συναλλαγή.

\subsection{Η Έννοια της Φήμης}\label{sec:rep}

Γυρίζοντας πίσω στον κόσμο τον ανθρώπων μπορούμε εύκολα να εντοπίσουμε άλλη μία έννοια η οποία χαρακτηρίζει εάν μεγάλο αριθμό ενεργειών και αποφάσεων μας. Αυτή η έννοια είναι η φήμη. Όταν για παράδειγμα θέλει ένα άτομο να αποκτήσει πρόσβαση σε μία υπηρεσία καινούργια (δηλαδή που δεν την έχει ξαναχρησιμοποιήσει), όπως για παράδειγμα να πάει σε ένα φροντιστήριο το παιδί του να μάθει μία ξένη γλώσσα ή να βρει έναν καλό μηχανικό για να του χτίσει το σπίτι του. Τότε ένα από τους βασικούς παράγοντες της επιλογής θα είναι η φήμη. Δηλαδή θα συλλέξει πληροφορίες από τον κοινωνικό του περίγυρο και αφού τις σταθμίσει ανάλογα με την εμπιστοσύνη του στης πηγές προέλευσής τους θα εξάγει την φήμη του πιθανού παρόχου της υπηρεσίας.
\newpage

Στην Υπολογιστική Επιστήμη μοντελοποιούμε την φήμη με ακριβώς τον ίδιο τρόπο. Έτσι μπορούμε να ορίσουμε την φήμη ως:
\begin{reputation}
Φήμη είναι το τι λέγεται ή πιστεύεται δημόσια για το ποιόν ή τον χαρακτήρα κάποιου ατόμου ή αντικειμένου
\end{reputation}

Παρ'όλη την στενή συγγένεια των εννοιών της φήμης και της εμπιστοσύνης, έχουν μία θεμελιώδη διαφορά. Η φήμη βασίζεται στην εμπειρία του κοινωνικού συνόλου με ένα άτομο ενώ η εμπιστοσύνη είναι προσωπικό ζήτημα και δεν επηρεάζεται από το κοινωνικό σύνολο. Μπορούμε να δούμε ξεκάθαρα την διαφορά τους στης παρακάτω προτάσεις:
\begin{enumerate}
\item \begin{quote} 
"Θα εμπιστευτώ τον Χ επειδή έχει καλή φήμη"
\end{quote}
\item \begin{quote}
 "Θα εμπιστευτώ τον Υ παρόλο που έχει κακή φήμη, επειδή μαζί μου είναι καλός"
\end{quote}
\end{enumerate}

Οι παραπάνω φράσεις δείχνουν ξεκάθαρα πώς παρόλο που η υπηρεσία που τα δυο άτομα θέλουν αιτηθούν είναι η ίδια, το άτομο 1, που δεν έχει αλληλεπιδράσει ποτέ με τον Υ, θα προτιμήσει τον Χ εμπιστευόμενος το κοινωνικό σύνολο από όπου εξήγε την καλή φήμη του Χ. Αντίθετα το άτομο 2 ένα και ξέρει πως ο Χ έχει καλύτερη φήμη από τον Υ θα διαλέξει τον Υ επειδή έχει καλή προϊστορία μαζί του, ουσιαστικά επειδή τον εμπιστεύεται. Αυτή η παρατήρηση δείχνει πώς η υποκειμενική γνώμη ενός ατόμου για ένα άλλο έχει μεγαλύτερο βάρος από την γνώμη του κοινωνικού σύνολου και έτσι στο τέλος η εμπιστοσύνη σε ένα άτομο θα υπερισχύσει της φήμης του.

\section{Αρχιτεκτονικές Δικτύωσης Συστημάτων Εμπιστοσύνης}

Όπως αναφέραμε οι κλασσικοί τρόποι που έχει η κοινωνία για την εξαγωγή φήμης και εμπιστοσύνης απουσιάζουν από τα online συστήματα που κατασκευάζουμε. Για αυτό τον λόγο χρειαζόμαστε ηλεκτρονικά υποκατάστατα. Οι βασικότερες ιδιότητες που πρέπει να έχει ένα ένα σύστημα βασιζόμενο στην φήμη είνα: %\citep{Resnick} απο το ίδιο

\begin{enumerate}
\item Οι οντότητες πρέπει να είναι μακρόβιες έτσι ώστε μετά από μία συναλλαγή να υπάρχει πιθανότητα μία μελλοντικής επαφής με την ίδια οντότητα

\item Οι συναλλαγές να αποτιμώνται και οι βαθμολογίες να διανέμονται σε άλλα μέλη τις κοινότητας

\item Οι αποφάσεις για της συναλλαγές μίας οντότητας να επηρεάζονται από τις βαθμολογίες των παλιών συναλλαγών
\end{enumerate}

Οι τεχνικές αρχές για τα συστήματα φήμης περιγράφονται στο παρών και το ακόλουθο τμήμα. Η αρχιτεκτονική του δικτύου καθορίζει πώς οι αξιολογήσεις των συναλλαγών αξιοποιούνται και διανέμονται ανάμεσα στα μέλη των συστημάτων. Οι δύο κύριοι τύποι είναι οι συγκεντρωτικές και οι κατανεμημένες αρχιτεκτονικές.%\cite pali ton josang
\subsection{Συγκεντρωτικές Αρχιτεκτονικές}

Στις συγκεντρωτικές αρχιτεκτονικές, οι πληροφορίες για την επίδοση ενός μέλους συγκεντρώνονται από τις βαθμολογίες των άλλων μελών της κοινότητας οι οποίοι είχαν άμεση εμπειρία με τον πρώτο. Η κεντρική αρχή η οποία συγκεντρώνει τις βαθμολογίες εξάγει τελικά την δείκτη της φήμης του κάθε μέλους της κοινότητας τον οποίο μπορεί να δει όποιος ενδιαφέρεται. Τα άλλα μέλη με την σειρά του χρησιμοποιούν αυτούς τους δείκτες όταν θέλουν να αλληλεπιδράσουν με άγνωστα, για αυτούς, μέλη της κοινότητας και θέλουν να αποφασίσουν εάν θα προβούν σε κάποια συναλλαγή ή όχι. Η ουσία είναι πώς συναλλαγές με άτομα που έχουνε υψηλότερο δείκτη φήμης έχουν πιο πολλές πιθανότητες να είναι επιτυχημένες από συναλλαγές με άτομα χαμηλού δείκτη φήμης.

Στο Σχ.\ref{fig:centralized.png} παρακάτω φαίνεται μία τυπική αρχιτεκτονική με κεντρική αρχή.

%
\diagramscale{Συγκεντρωτική Αρχιτεκτονική}{centralized.png}{0.9}

Ο Α θέλει να προβεί σε μία συναλλαγή με τον Β, ενώ δεν έχει κάποια παρελθοντική εμπειρία με αυτόν. Για αυτό ρωτάει την κεντρική αρχή, η οποία με την σειρά της γυρίζει τον δείκτη φήμης που έχει υπολογίσει βασιζόμενη σε πληροφορίες από άλλα μέλη τις κοινότητας όταν αυτά αλληλεπίδρασαν με τον Β και κοινοποίησαν στην κεντρική αρχή την εμπειρία τους.

Τα δύο βασικά κομμάτια των συγκεντρωτικών συστημάτων φήμης είναι:
\begin{enumerate}

\item Tα συγκεντρωτικά πρωτόκολλα επικοινωνίας που επιτρέπουν στους συμμετέχοντες να παρέχουν αξιολογήσεις για τα μέλη των συναλλαγών στην κεντρική αρχή, καθώς και να αποκτήσουν έναν δείκτη φήμης των πιθανών εταίρων στην συναλλαγή από την κεντρική εξουσία.

\item Έναν τρόπο υπολογισμού φήμης που χρησιμοποιείται από την κεντρική αρχή να εξάγει δείκτες φήμη για κάθε μέλος, με βάση τις αξιολογήσεις που έλαβε, και ενδεχομένως άλλες πληροφορίες όπως θα δούμε στην παράγραφο \ref{sec:compute} παρακάτω.

\end{enumerate}
\newpage
\subsection{Κατανεμημένες Αρχιτεκτονικές}

Υπάρχουν περιβάλλοντα όπου ένα κατανεμημένο σύστημα φήμης, δηλαδή χωρίς τις κεντρικές λειτουργίες, είναι καταλληλότερο από ένα συγκεντρωτικό σύστημα. Σε ένα κατανεμημένο σύστημα, δεν υπάρχει κεντρικό σημείο για την υποβολή των αξιολογήσεων ή την απόκτηση δεικτών φήμης των άλλων. Αντ'αυτού, μπορεί να υπάρχουν κατανεμημένα σημεία αποθήκευσης όπου μπορούν να υποβληθούν αξιολογήσεις ή ο κάθε συμμετέχοντας απλά να καταγράφει τη γνώμη του σχετικά με τις εμπειρίες του με άλλα μέλη της κοινότητας, και να παρέχει ο ίδιο τις πληροφορίες αυτές, κατόπιν αιτήματος. Ένας τρίτος συμβαλλόμενος, ο οποίος ενδιαφέρεται να συναλλαχθεί με δεδομένο άλλο μέλος της κοινότητας, πρέπει να βρει τις κατανεμημένες αποθήκες, ή να προσπαθήσει να αποκτήσει βαθμολογίες από όσα μέλη της κοινότητας μπορεί τα οποία είχαν άμεση εμπειρία με το εν λόγω μέλος-στόχο. Αυτό απεικονίζεται στο Σχ.\ref{fig:distributed.png}  παρακάτω.

\diagramscale{Κατανεμημένη Αρχιτεκτονική}{distributed.png}{0.9}

Ο τρίτος συμβαλλόμενος (στο Σχ.\ref{fig:distributed.png}  ο Α) υπολογίζει μόνος του των δείκτη της φήμης του μέλους στόχου (Β) βασιζόμενος στην μερική ή ολική γνώση που συνέλεξες από ένα τους "γνωστούς" του. (εδώ έχει μερική γνώση επειδή δεν πήρε feedbacκ από τον Δ).

Τα δύο βασικά κομμάτια των κατανεμημένων συστημάτων φήμης είναι:
\begin{enumerate}

\item Tα κατανεμημένα πρωτόκολλα επικοινωνίας που επιτρέπουν στους συμμετέχοντες να λάβουν και να στείλουν βαθμολογίες για άλλα μέλη της κοινότητας

\item Έναν τρόπο υπολογισμού φήμης που χρησιμοποιείται από κάθε μέλος της κοινότητας για να εξάγει δείκτες φήμης με βάση τις αξιολογήσεις που έλαβε, και ενδεχομένως άλλες πληροφορίες όπως θα δούμε στην παράγραφο \ref{sec:compute} παρακάτω.

\end{enumerate}
\newpage
\section{Τρόποι υπολογισμοί φήμης και εμπιστοσύνης}\label{sec:compute}

\subsection{Bayesian}

Τα Bayesian συστήματα λαμβάνουν δυαδικές αξιολογήσεις ως είσοδο (δηλαδή θετική ή αρνητική τιμή), και βασίζονται στον υπολογισμό των δεικτών  με χρήση των \textit{βήτα συναρτήσεων πυκνότητας πιθανότητας} (ΣΠΠ). Ο  ενημερωμένος δείκτης υπολογίζεται συνδυάζοντας τις προηγούμενες τιμές των δεικτών φήμης με τη νέα βαθμολόγια.%\cite apo johang [31,49-51,68].
 Ο δείκτης της φήμης ή/και της εμπιστοσύνης μπορεί να αναπαρασταθεί είτε με τη μορφή μιας πλειάδας παραμέτρων βήτα μορφής \textit{(α, β)} (όπου \textit{α} και \textit{β} αντιπροσωπεύουν τον αριθμό των θετικών και αρνητικών αξιολογήσεων, αντίστοιχα), είτε με τη μορφή της τιμής της προσδοκία της βήτα ΣΠΠ. Προαιρετικά μπορεί να υπάρχει και ή διακύμανση. Το πλεονέκτημα των Bayesian συστημάτων είναι ότι παρέχουν μία θεωρητικά ορθή βάση για τον υπολογισμό των δεικτών, ενώ το μόνο μειονέκτημα που θα μπορούσε να καταλογιστεί είναι ότι είναι υπερβολικά πολύπλοκα για να τα κατανοήσει ο μέσος άνθρωπος.

Η βήτα οικογένεια κατανομών είναι μια συνεχής οικογένεια συναρτήσεων κατανομής μεταβαλλόμενη από τις δύο παραμέτρους α και β. Η βήτα ΣΣΠ συμβολίζεται με beta(p | α, β) μπορεί να εκφραστεί χρησιμοποιώντας τη συνάρτηση γάμμα Γ ως εξής:
\begin{equation}
beta(p|\alpha,\beta)= \frac{\Gamma\left(\alpha + \beta\right)}{\Gamma\left(\alpha\right)\Gamma\left(\beta\right)}p^{\alpha−1}\left(1-p\right)^{\beta-1} o \pi o \upsilon\ \  0\  ≤\  p\  ≤\  1,\  \alpha,\  \beta\  >\  0
\label{eq:beta}
\end{equation}

με τον περιορισμό ότι η πιθανότητα $ p \neq 0 \ \  \alpha \nu \  \alpha \  <\ 1,$ και $ p \neq 1 \ \  \alpha \nu \  \beta \  <\ 1$.
Η τιμή της προσδοκίας της κατανομής beta δίνεται από:

\begin{equation}
E\left(p\right) = \frac{\alpha}{\left(\alpha+\beta\right)}
\label{eq:exp}
\end{equation}

Στην αρχική κατάσταση όπου δεν είναι τίποτα γνωστό η εκ των προτέρων ΣΠΠ είναι η ομοιόμορφη κατανομή βήτα κατανομή όπου $\alpha=1,\ \beta = 1$ η οποία φαίνεται στο σχήμα \ref{fig:beta.png}  παρακάτω. Στην συνέχεια αφού παρατηρηθούν ν θετικές και μ αρνητικές βαθμολογίες η εξαγόμενη κατανομή είναι η βήτα κατανομή(α,β) όπου α=ν+1 και β=μ+1. Για παράδειγμα εάν παρατηρηθούν 7 θετικές και 1 αρνητική βαθμολογίες τότε η ΣΠΠ φαίνεται στο σχήμα \ref{fig:beta.png} .


\diagramscale{Βήτα Συναρτήσεις Πυκνότητας Πιθανότητας}{beta.png}{0.4}
Μία ΣΠΠ αυτού του τύπου εκφράζει την αβέβαιη πιθανότητα ότι οι μελλοντικές αλληλεπιδράσεις θα είναι θετικές. Το πιο φυσικό είναι να καθοριστεί ο δείκτης φήμης ως συνάρτηση της τιμής της προσδοκίας. Η τιμή αυτή (της πιθανότητας προσδοκίας) σύμφωνα με την Εξ. \ref{eq:exp} είναι Ε(p) = 0,8. Αυτό μπορεί να ερμηνευθεί λέγοντας ότι η σχετική συχνότητα μιας θετικής έκβασης στο μέλλον είναι αβέβαιη, και ότι ή πιθανότερη τιμή της (της συχνότητας) είναι 0,8.

\subsection{ Διακριτών Καταστάσεων}

Οι άνθρωποι μπορούν συχνά να αξιολογήσουν καλύτερα την απόδοση συστημάτων όταν οι τιμές έχουν διακριτή λεκτική μορφή, σε σχέση με όταν είναι συνεχείς μετρήσεις. Αυτό ισχύει επίσης και για τον καθορισμό των δεικτών εμπιστοσύνης. Ορισμένοι συγγραφείς, συμπεριλαμβανομένων %\cite[1,8,9,44],
 έχουν προτείνει διακριτά μοντέλα εμπιστοσύνης. Για παράδειγμα, στο μοντέλο του Αμπντούλ Ραχμάν \& Hailes (2000) %\cite[1]
 , η αξιοπιστία ενός παράγοντα Χ μπορεί να έχει τιμές:
Πολύ Αξιόπιστη, Αξιόπιστη, Αναξιόπιστη, Πολύ Αναξιόπιστη
Το μέλος της κοινότητας (Υ) που θέλει να βρει την αξιοπιστία κάποιου μπορεί να συνδυάσει τη δική του άποψη  για την αξιοπιστία του Χ με την φήμη που εξάγει από την κοινότητα.
 
Look-up πίνακες, με εγγραφές για την υποβάθμιση / αναβάθμιση της εμπιστοσύνης που έχει ο Υ στον Χ, χρησιμοποιούνται για να καθορίσουν την εξαγόμενη εμπιστοσύνη στο Χ. Κάθε φορά που ο Υ είχε προσωπική εμπειρία με το Χ, αυτό μπορεί να χρησιμοποιηθεί για τον προσδιορισμό τής αξιοπιστίας του. Η υπόθεση είναι ότι η προσωπική εμπειρία αντικατοπτρίζει την πραγματική αξιοπιστία του Χ και ότι η φήμη που λαμβάνει για τον Χ, η οποία διαφέρει από την προσωπική εμπειρία, δείχνει αν ο Υ υποτιμά ή υπερτιμά τον Χ. Παρατηρήσεις από "φίλους" που υπερεκτιμούν των Υ θα υποβαθμιστεί πριν ληφθούν υπόψιν, και το αντίστροφο.

Το μειονέκτημα των διακριτών τιμών είναι ότι δεν είναι εύκολο να εφαρμοστούν σε αυτά κλασσικοί υπολογισμοί. Αντ 'αυτού, πρέπει να χρησιμοποιηθούν ευριστικοί μηχανισμοί όπως look-up πίνακες.

\subsection{Ασαφούς Λογικής}

Η εμπιστοσύνη και η φήμη μπορούν να παρασταθούν ως γλωσσικά ασαφείς έννοιες, όπου συναρτήσεις συμμετοχής περιγράφουν σε ποιο βαθμό ένας παράγοντας μπορεί να περιγραφεί ως π.χ. αξιόπιστος ή αναξιόπιστες. Η ασαφής λογική προβλέπει κανόνες για το συλλογισμό με ασαφείς μετρικές αυτού του είδους. Το σύστημα που προτείνει Manchala (1988%\cite [44]
,καθώς και η εξαγωγή τής φήμη στο συστήματος REGRET που προτείνουν οί Sabater \& Sierra (2001,2002) %\cite [59-61]
 εμπίπτουν σε αυτή την κατηγορία. Στο σύστημα των Sabater \& Sierra, αυτό που αποκαλούν ατομική φήμη προέρχεται από ιδιωτικές πληροφορίες σχετικά με ένα συγκεκριμένο δράστη ενώ αυτό που ονομάζουν κοινωνική φήμη προέρχεται από την ενημέρωση του κοινού σχετικά με αυτόν τον δράστη, τέλος αυτό που αποκαλούν context dependent φήμη εξάγεται από τις συμφραζόμενες πληροφορίες.

\subsection{Ροής Εμπιστοσύνης}

Τα συστήματα που υπολογίζουν την εμπιστοσύνη ή τη φήμη με επανάληψη μέσω βρόχου ή με αυθαίρετα μεγάλες αλυσίδες μπορούν να ονομαστούν μοντέλα ροής.

Ορισμένα μοντέλα ροής έχουν σταθερή τιμή ολικής εμπιστοσύνης / φήμης για το σύνολο του συστήματος και η τιμή αυτή μπορεί να κατανεμηθεί μεταξύ των μελών της κοινότητας. Οι συμμετέχοντες μπορούν να αυξήσουν την εμπιστοσύνη / φήμη τους μόνο σε βάρος των άλλων. Ό PageRank της Google [52] ανήκει σε αυτή την κατηγορία. Σε γενικές γραμμές, η φήμη ενός συμμετέχοντος αυξάνει ως συνάρτηση της εισερχόμενης ροής, και μειώνεται ως συνάρτηση της εξερχόμενης ροής. Στην περίπτωση της Google, πολλοί υπερσύνδεσμοι εισερχόμενοι (που δείχνουν σε αυτή) σε μια ιστοσελίδα συμβάλλουν στην αύξηση του δείκτη PageRank ενώ πολλοί υπερσύνδεσμοι εξερχόμενοι (που υπάρχουν μέσα) από μια ιστοσελίδα οδηγούν στη μείωση του δείκτη PageRank για την εν λόγω ιστοσελίδα.

Τα μοντέλα ροής δεν απαιτούν πάντα το άθροισμα των βαθμολογιών φήμης / εμπιστοσύνης να είναι σταθερό. Ένα τέτοιο παράδειγμα είναι το σύστημα EigenTrust [35], το οποία υπολογίζει δείκτες εμπιστοσύνη σε δίκτυα P2P μέσω επαναλαμβανόμενου πολλαπλασιασμού και  ομαδοποίησης των βαθμολογιών εμπιστοσύνης έως ότου οι δείκτες εμπιστοσύνης για όλα τα μέλη της P2P κοινότητας συγκλίνουν σε σταθερές τιμές.

\subsection{Άθροιση ή εξαγωγή Μέσου Όρου}

Η συνηθέστερη μορφή υπολογισμού φήμης / εμπιστοσύνης είναι να υπολογίζεται το συνολικό αριθμό των θετικών αξιολογήσεων και το συνολικό αριθμό των αρνητικών αξιολογήσεις ξεχωριστά, και να εξάγετε μια συνολική βαθμολογία ως το θετικό άθροισμα μείον το αρνητικό. Αυτή είναι η αρχή που χρησιμοποιείται στο φόρουμ του eBay η οποία περιγράφεται λεπτομερώς στο [55]. Το πλεονέκτημα είναι ότι ο καθένας μπορεί να καταλάβει την αρχή πίσω τον υπολογισμό της φήμη, το μειονέκτημα είναι ότι ο υπολογισμός είναι πρωτόγονος και, ως εκ τούτου ο δείκτης της φήμης δίνει μιας φτωχής εικόνα αν και αυτό οφείλεται επίσης στον τρόπο που παρέχεται αξιολόγηση.

Ένα ελαφρώς πιο προχωρημένο σύστημα που προτείνεται στο π.χ. [63] είναι να υπολογιστεί ο δείκτης της φήμη ως ο μέσος όρος όλων των αξιολογήσεων. Η αρχή αυτή χρησιμοποιείται στα συστήματα φήμης πολλών εμπορικές ιστοσελίδων, όπως  της Amazon περιγράφεται στο \ref{sec:prev}

Προηγμένα μοντέλα σε αυτή την κατηγορία υπολογίζουν ένα σταθμισμένο μέσο όρο όλων των αξιολογήσεων, όπου το βάρος της αξιολόγησης μπορεί να καθορίζεται από παράγοντες όπως ο εκτιμητής της αξιοπιστίας / φήμη, τη ηλικία της αξιολόγησης, η σχέση μεταξύ αξιολογητή και αξιολογούμενου κ.λ.π. Το σύστημα που παρουσιάζεται στην υπόλοιπη εργασία εμπίπτει σε αυτήν την κατηγορία.
\newpage
\section{Γνωστά Συστήματα φήμης κ εμπιστοσύνης σε διαφόρους τομείς}\label{sec:prev}

Τα συστήματα φήμης και εμπιστοσύνης είναι ευρέως διαδεδομένα σε πολλούς τομείς όπου χρησιμοποιούνται πληροφοριακά συστήματα. Ο βασικός λόγος που εφαρμόζεται ένα σύστημα φήμης-εμπιστοσύνης είναι για να αντιμετωπίσει κακόβουλους συμμετέχοντες σε μία κοινότητα οντοτήτων και έτσι να βελτιωθεί τόσο η ασφάλεια όσο και η ποιότητα των υπηρεσιών.

\subsection{Business}

\subsubsection{E-bay}

\subsubsection{Amazon}

\subsubsection{PageRank}

\subsection{Mobile-ad-hoc Networks}

\subsubsection{Le Boudec}

\subsubsection{hubeaux}

\subsubsection{stanford}

\subsection{Peer-to-Peer Systems}

\subsubsection{EingenTrust}

\subsubsection{PeerTrust}

\subsubsection{PowerTrust}

\subsection{Internet of Things}

\subsubsection{Atzori}


