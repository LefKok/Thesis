\chapter{Θεωρητικό και Τεχνολογικό Υπόβαθρο}\label{ch:bkg}
\section{Ορισμοί Εννοιών}

Τα συστήματα εμπιστοσύνης και φήμης αποτελούν μία εξέλιξη της τελευταίας δεκαπενταετίας και χρησιμοποιούνται ευρέως σε διαδικτυακές αλληλεπιδράσεις μεταξύ τόσο ανθρώπων (e-bay,amazon) όσων και μηχανών (ευφυών δραστών, peer-to-peer συστημάτων κτλ.). Η ιδέα πάνω στην οποία βασίζονται είναι: η συλλογή κριτικής για έναν δράστη από άλλα μέλη της κοινότητας τα οποία έχουν ήδη αποκτήσει κάποιες εμπειρίες με τον πρώτο. Με αυτόν τον τρόπο υπάρχει η δυνατότητα να γνωρίζει κάποιος για "το ποιόν"  κάποιου άλλου χωρίς να χρειάζεται να τον γνωρίσει, δηλαδή να αλληλεπιδράσει με αυτόν. Αυτή η δυνατότητα με την σειρά της "εξαναγκάζει" του δράστες ενός συστήματος να συμμορφωθούν με τους κανόνες της δικτυακή κοινότητας της οποίας είναι μέλη.

Όταν η αίτηση για παροχή υπηρεσιών ή για απόκτηση δεδομένων γίνεται online μεταξύ οντοτήτων πρότερα αγνώστων μεταξύ τους τότε μπορούν να προκύψουν 2 βασικά προβλήματα.
\begin{enumerate}
\item Το πρώτο πρόβλημα που δέν εξετάζεται στην παρούσα εργασία είναι πώς τα δεδομένα που παρέχονται (και στην περίπτωση του COSMOS η γνώση) μπορεί να χρησιμοποιηθούν από τον παραλήπτη για διαφορετικούς σκοπούς από αυτούς που ισχυρίζεται. Έτσι να τον βοηθήσουμε άθελα μας στου κακόβουλους ή γενικότερα αντίθετους από το καλό της κοινότητας σκοπούς.

Για αυτό θα πρέπει να υπάρχει ένας μηχανισμός παρακολούθησης των ενεργειών του παραλήπτη μετά την απόκτηση των πληροφοριών κάτι που στην ψυχολογία αναφέρεται ώς 

\item Αντίθετα αυτή η εργασία εστιάζει στο πρόβλημα του ρίσκου πού παίρνει ένας χρήστης όταν ζητάει μία υπηρεσία, επειδή μπορεί τελικά να μην την λάβει σε ικανοποιητικό βαθμό οπότε να πρέπει να την ξαναζητήσει. Ακόμα είναι δυνατόν να κινδυνέψει η ασφάλεια του (αντί για σωστό αρχείο να σταλεί κάτι που περιέχει ένα Trojan), ενώ στην δική μας περίπτωση μπορεί η γνώση και εν συνεχεία οι δράσεις που προτείνονται να προκαλέσουν προβλήματα ασφαλείας όχι μόνο στον ψηφιακό κόσμο  των Εικονικών Οντοτήτων, όσο και στο φυσικό κόσμο των πραγμάτων και των ανθρώπων.
\end{enumerate}

\subsection{Η Έννοια της Εμπιστοσύνης}

Στον κόσμο των ανθρώπινων οντοτήτων η ύπαρξη της έννοιας τις εμπιστοσύνης είναι εμφανής στις καθημερινές μας αλληλεπιδράσεις. Θα μιλήσουμε για τα προβλήματα μας και της εμπειρίες μας -δηλαδή θα δώσουμε ευαίσθητες προσωπικές πληροφορίες- στους φίλους μας οι οποίοι περιμένουμε να μας βοηθήσουνε με δικές τους εμπειρίες, καθώς και να βοηθηθούν από τις πληροφορίες μας. Επίσης θα πάμε να αγοράσουμε προϊόντα από μαγαζιά που γνωρίζουμε και έχουμε ξαναπάει επειδή πιστεύουμε πως η ποιότητα ου προϊόντος ή της υπηρεσίας που θα λάβουμε θα είναι ικανοποιητική για εμάς.

Στον κόσμο όμως της υπολογιστικής επιστήμης ο ορισμός και η μοντελοποίηση της εμπιστοσύνης είναι αρκετά δύσκολη. Αυτό συμβαίνει επειδή τις ανθρώπινες σχέσεις τις κυβερνούν συναισθήματα όπως αγάπη και αφοσίωση που δεν έχουν καμία θέση στον ψηφιακό κόσμο. Στο έργο του Audun Jøsang %\cite survey trust josang
υπάρχει μία εξαιρετική συγκέντρωση για τους διαφόρους ορισμούς της εμπιστοσύνης και τις φήμης.
Εκεί λοιπόν αναφέρεται πώς υπάρχουν δύο βασικοί ορισμοί για την εμπιστοσύνη τους οποίους ονομάζουμε εμπιστοσύνη αξιοπιστίας (reliability trust) και εμπιστοσύνη απόφασης (decision trust)


Όπως προϊδεάζει και το όνομα η εμπιστοσύνη αξιοπιστίας μπορεί να ερμηνευθεί ως η αξιοπιστία κάποιο δράστη ή αντικειμένου, και ο ορισμός του Gambetta%\cite gambetta apo josang
παρέχει ένα παράδειγμα του πώς μπορεί να διατυπωθεί:
\begin{reliability}
Εμπιστοσύνη είναι η υποκειμενική πιθανότητα από την οποία ένα άτομο, Α, περιμένει από ένα άλλο άτομο, Β, να κάνει μία συγκεκριμένη ενέργεια από την οποία εξαρτάται η ευημερία του.
\end{reliability}

Αυτός ορισμός περιεχέι την έννοια της εξάρτησης του Α από τον Β , και την αξιοπιστία (δηλαδή την πιθανότητα) του Β όπως την αντιλαμβάνεται ο Α.

Όμως η εμπιστοσύνη μπορεί να είναι πολύ πιο σύνθετη από τον παραπάνω ορισμό. Αυτό συμβαίνει επειδή το γεγονός ότι η αξιοπιστία ενός δράστη είναι υψηλή δεν οδηγεί στην τυφλή εμπιστοσύνη του για οποιαδήποτε ενέργεια, επειδή το ρίσκο τις αποτυχίας μίας συγκεκριμένη συναλλαγής μπορεί να είναι πολύ μεγάλο.Δηλαδή με απλά λόγια, ένας δράστης μπορεί να θεωρηθεί αξιόπιστος για μία απλή συναλλαγή, όμως για κάποια πιο κρίσιμη η αξιοπιστία του να μην είναι αρκετά καθησυχαστική. Για να συμπεριλάβουμε λοιπόν την πιο ευρεία έννοια που μπορεί να χαρακτηριστεί ως εμπιστοσύνη, μπορεί να χρησιμοποιηθεί ο ακόλουθος ορισμός των  McKnight \& Chervany %\cite mcknight apo josang

\begin{decision}
Η εμπιστοσύνη είναι ο βαθμός στον οποίο ένα άτομο είναι πρόθυμο να εξαρτηθεί από κάτι ή κάποιον σε μια δεδομένη κατάσταση με ένα αίσθημα σχετικής ασφάλειας, ακόμη και αν οι αρνητικές συνέπειες είναι δυνατές.
\end{decision}

Ο γενικότερος αυτός ορισμός μας παρέχει μία χρήσιμη ασάφεια αφού περιλαμβάνει πτυχές μίας ευρύτερης έννοιας της εμπιστοσύνης,η οποία περιέχει της έννοιες της εξάρτησης από αυτόν που απολαμβάνει την εμπιστοσύνη, της αξιοπιστίας αυτού καθώς και της ύπαρξης ενός παράγοντα κινδύνου που πρέπει να αποδεχτεί το άτομο που καλείτε να εμπιστευτεί κάποιο άλλο άτομο, ανάλογο με την εκάστοτε συναλλαγή.

\subsection{Η Έννοια της Φήμης}