\chapter{ Επίλογος και Μελλοντικές Επεκτάσεις}\label{ch:conclusion}


Στην ενότητα αυτή γίνεται μία σύνοψη της εργασίας και παρουσιάζονται τα
συνολικά συμπεράσματα που μπορούμε να βγάλουμε από τα αποτελέσματα.
Γίνεται αναφορά στα θετικά και τα αρνητικά στοιχεία του συστήματός μας,
όπως αυτά προκύπτουν από τις μετρήσεις μας, ενώ παρουσιάζονται και ιδέες
για επέκταση του συστήματος, βελτιώσεις και αντιμετώπιση των αδύναμων
σημείων του.


\section{Σύνοψη και συμπεράσματα}

Στόχος της Εργασίας ήταν η δημιουργία ενός συστήματος εμπιστοσύνης φήμης το οποίο θα είναι ικανό να παρέχει υψηλή ικανοποίηση στους δράστες μίας ψηφιακής κοινότητας και συγκεκριμένα στης Εικονικές Οντότητες του CosmoS. Μία από τις βασικές σχεδιαστικές απαιτήσεις ήταν η ικανότητα να υπάρχει υποστήριξη χιλιάδων δραστών χωρίς την ύπαρξη μεγάλου υπολογιστικού φόρτου. Αυτό επιτεύχθηκε μη την μοντελοποίηση της ανθρώπινης κοινωνίας σε θέματα επιλογής παροχών υπηρεσιών όπου καταφέρνει να λειτουργήσει αποδοτικά με ελάχιστη γνώση του κοινωνικού συνόλου για της δράσεις κάθε ατόμου. 

Δεν εγγυόμαστε η προτάσεις που παρέχει το σύστημα να είναι οι βέλτιστες αλλά εξασφαλίζουμε υψηλή ικανοποίηση κάθε χρήστη η οποία μπορεί εύκολα να εξατομικευτεί ρυθμίζοντας κατάλληλα το βάρος μίας συναλλαγής και το κατώφλι ικανοποίησης για την επόμενη.

Όπως φάνηκε και από τις μετρήσεις η ικανοποίηση που παρέχεται από το RT-IOT είναι συγκρίσιμη με τους βασικότερους αλγορίθμους που εφαρμόζονται σήμερα ενώ υπάρχει η πεποίθηση ότι μπορεί να υποστηρίξει πολύ μεγαλύτερο αριθμό δραστών. Για αυτό ο κώδικας έχει είδη ενσωματωθεί στις Εικονικές Οντότητες του CosmoS.

\section{Μελλοντικές Επεκτάσεις}

To πρώτο επόμενο βήμα είναι η αξιολόγηση του συστήματος σε πραγματικές συνθήκες μέσα από το CosmoS. Άλλα χαρακτηριστικά που θα μπορούσαν να ενσωματωθούν στο σύστημα είναι αρχικά, στο επίπεδο προσομοίωσης η καλύτερη αναπαράσταση της διαδικασίας ανταλλαγής γνώσης επειδή πρέπει να υπάρξει και ένας δεύτερος παράγοντας πέραν της εμπιστοσύνης που ονομάζεται similarity. Αυτός ο παράγοντας δείχνει πόσο κοντά στο πρόβλημα της εικονικής οντότητας είναι ο σενάριο που της παρέχεται. Έτσι το πώς θα ρυθμιστεί το κατώφλι είναι συνάρτηση του πόσο κοντά στο δικό της πρόβλημα είναι η πρόταση. Για να συμβεί όμως αυτό πρέπει να αντιμετωπιστεί η περίπτωση κακόβουλης παροχής γνώσης φαινομενικά μεγάλης ομοιότητας που θα οδηγούσε σε χαμηλότερο κατώφλι.

Άλλες ιδέες είναι η ομαδοποίησης των εικονικών οντοτήτων με βάση τον χρήστη. Δηλαδή η δημιουργία ιεραρχίας στο σύστημα. Έτσι μπορούν πολλές συσκευές του ίδιου χρήστη να εμφανίζονται στο σύστημα ως μία Εικονική Οντότητα με πολλές υπηρεσίες. Αυτό θα αύξανε περαιτέρω την πραγματική κλιμάκωση ενώ θα έδινε την δυνατότητα σε υπολογιστικά αδύναμες συσκευές να χρησιμοποιήσουν το σύστημα. 