\chapter{Τύποι επιθέσεων - Αντιμετώπιση με RT-IOT}\label{ch:design}

\section{Επιθέσεις στο Διαδίκτυο των πραγμάτων}

Όπως αναφέρθηκε στο Κεφάλαιο \ref{ch:bkg} ένας από τους κινδύνους παραβίασης της ασφάλειας του Κοινωνικού Διαδικτύου των Πραγμάτων πηγάζει από επιθέσεις κακόβουλων οντοτήτων οι οποίες όμως νομίμως δρουν εντός της κοινωνίας των πραγμάτων. Σε αυτό το κομμάτι του κεφαλαίου θα αναλύσουμε τις βασικότερες επιθέσεις που μπορούν να συμβούν.

\subsection{Μεμονωμένες κακόβουλες Εικονικές Οντότητες}\label{sec:indv}

Οι μεμονωμένες κακόβουλες Εικονικές Οντότητες Σχ.\ref{fig:individual.png} είναι o απλούστερος και ο βασικότερος τύπος επίθεσης. Σε αυτήν την περίπτωση ένα ποσοστό των μελών της κοινότητας έχοντας κακόβουλους σκοπούς προσπαθούν να εκμεταλλευτούν την εμπιστοσύνη των υπόλοιπων και όταν τους ζητηθεί κάποια υπηρεσία ή γνώση αντί να τους εξυπηρετήσουν σωστά στέλνουν δεδομένα που θα οδηγήσουν στο να εκτεθεί ο παραλήπτης σε κίνδυνο ή να οδηγηθεί σε δυσλειτουργία. 

\diagramscale{Μεμονωμένες κακόβουλες Εικονικές Οντότητες}{individual.png}{0.45}
\newpage
Η ύπαρξη κακόβουλων Εικονικών Οντοτήτων αποτελεί την βάση όλων των επιθέσεων και ο εντοπισμός τους είναι ο κύριος στόχος του RT-IOT. Για να βρεθούν χρειάζεται η συλλογική συμφωνία του συστήματος στο ποιες ακριβώς είναι.

Αλλά ακόμα και αν δεν τους εντοπίσει το σύστημα ώστε να σταματήσει να τους προτείνει ως πιθανούς φίλους με καλή φήμη, τα μέλη μπορούν αυτόνομα να προστατευθούν. Έτσι μετά από έναν αριθμό αλληλεπιδράσεων μία Εικονική Οντότητα θα χάσει την όποια εμπιστοσύνη τους στον κακόβουλο δράστη και θα τον απορρίψει από τον κοινωνικό του περίγυρο. Το θέμα είναι πόσο γρήγορα θα το κάνει ώστε να περιορίσει του κινδύνους.



\subsection{Συνεργαζόμενες κακόβουλες Εικονικές Οντότητες}\label{sec:col}

Σε αυτού του είδους της επίθεσης Σχ. \ref{fig:collusion.png} υπάρχει ένας αριθμός κακόβουλων Εικονικών Οντοτήτων όπως περιγράφηκε στην παράγραφο \ref{sec:indv} οι οποίες όμως τώρα έχουν γνώση (μερική ή ολική) για την ύπαρξή και άλλον κακόβουλων Εικονικών Οντοτήτων. Με αυτή λοιπόν την γνώση συνεργάζονται μεταξύ τους και όταν στέλνουν, είτε στην κεντρική αρχή είτε σε άλλα μέλη της κοινότητας, πληροφορίες για να εξαχθεί ο δείκτης φήμης παραποιούν τα δεδομένα τους και να παρουσιάζουν τους κακόβουλους συνεργάτες τους ως φερέγγυους.

\diagramscale{Συνεργαζόμενες κακόβουλες Εικονικές Οντότητες}{collusion.png}{0.4}

Αυτού του είδους η επίθεση είναι πολύ πιο επικίνδυνη από την προηγούμενη. Αυτό συμβαίνει επειδή δεν είναι δυνατόν να βασιστούν οι Εικονικές Οντότητες στην καλή θέληση των μελών της κοινωνίας για να εξαχθεί η φήμη αφού υπάρχει μεγάλη πιθανότητα να ψεύδονται για τις εμπειρίες τους. Για την αντιμετώπιση του πρέπει: \begin{itemize}
\item Στην περίπτωση της κεντρικής αρχής να εντοπιστεί όλη η κοινότητα των κακόβουλων δραστών και να σταματήσει να λαμβάνεται υπόψιν η γνώμη τους κατά την εξαγωγή φήμης.

\item Στην περίπτωση συστημάτων χωρίς κεντρική αρχή πρέπει η κάθε Εικονική Οντότητα να εντοπίσει κάποιου έμπιστους φίλους των οποίων η πιθανότητα να ψεύδονται για τις εμπειρίες τους θα είναι μειωμένη σε σχέση με το σύνολο.
\end{itemize}

Μια εξελιγμένη μορφή αυτής τις επίθεσης είναι η κακόβουλες Εικονικές Οντότητες να προσπαθούν να εντοπίσουν μέσω τις φήμης ανταγωνιστικές προς τις υπηρεσίες που εκμεταλλεύονται Εικονικές Οντότητες και να προσπαθήσουν να χαμηλώσουν την φήμη τους με περαιτέρω ψευδής δηλώσεις για την εμπειρία τους με αυτές.

\subsection{Μερικώς κακόβουλες Εικονικές Οντότητες}\label{sec:part}

Σε αυτού του είδους της επίθεσης Σχ. \ref{fig:partially.png} υπάρχουν κακόβουλες συνεργασίες όπως αναφέρθηκαν στην παράγραφο \ref{sec:col}. Η διαφορά τώρα όμως είναι πώς αυτές οι οντότητες προσπαθούν να νικήσουν το σύστημα φήμης αλλάζοντας την συμπεριφορά τους. Αυτό μπορεί αν συμβεί με 2 τρόπους: 
\begin{enumerate}

\item Αρχικά μπορούν να παρέχουν κακές υπηρεσίες μόνος $ p \ \%  $ των περιπτώσεων ώστε να μην καταποντίζεται η φήμη τους. Δηλαδή μόλις δουν ότι υπάρχει μία κίνηση αποκλεισμού τους, να γίνονται καλές ώστε να παραμείνουν ενσωματωμένες. Σε αυτή την περίπτωση πρέπει το σύστημα που υπολογίζει την φήμη να έχει μνήμη ώστε να βλέπει καχύποπτα τέτοιου είδους συμπεριφορές και να εφαρμόζει αποκλεισμό με το που ξαναγίνουν κακόβουλες. Παρόλα αυτά αυτή η επίθεση δεν είναι τόσο επικίνδυνη επειδή η κακή συμπεριφορά της Εικονικής Οντότητας είναι φραγμένη από το $p$

\item Ένας άλλος τρόπος να ξεγελάσουν το σύστημα είναι να προσθέσουν στο ενεργητικό τους και άλλες υπηρεσίες. Έτσι θα είναι πιο εύκολο να εκμεταλλεύονται μία υπηρεσία για τους κακόβουλους σκοπούς τους αλλά να παραμένουν στο απυρόβλητο από το σύστημα φήμης επειδή παρέχουν άλλες καλά. Για να αντιμετωπιστεί αυτού του είδους η επίθεση πρέπει να υπάρχει διαφοροποίηση των δεικτών φήμης ανά υπηρεσία. Αλλά για να μπορέσει παραμείνει κλιμακώσιμο πρέπει να γίνει ομαδοποίηση σε παρόμοιες υπηρεσίες ώστε να περιοριστεί το εύρος που μπορεί να εκμεταλλευτεί μία κακόβουλη Εικονική Οντότητα.
\end{enumerate}

\diagramscale{Μερικώς κακόβουλες Εικονικές Οντότητες}{partially.png}{0.35}
\newpage
\subsection{Κακόβουλοι κατάσκοποι}\label{sec:spies}

Αυτού του είδους η επίθεση  Σχ. \ref{fig:spies.png} είναι βοηθητική στις \ref{sec:col} και \ref{sec:part}. Οι Εικονικές Οντότητες - κατάσκοποι. Ονομάστηκαν έτσι επειδή στα μάτια του μεγαλύτερου μέρους των μελών του συστήματος παρουσιάζονται ως αξιόπιστες. Αυτό το καταφέρνουν παρέχοντας πάντα μία υπηρεσία καλά. Έτσι αυξάνουν την φήμη τους και την εμπιστοσύνη μεμονωμένων Εικονικών Οντοτήτων προς αυτές. Αυτό όμως που τις κάνει κακόβουλες είναι η συμπεριφορά τους όταν τους ζητηθεί αξιολόγηση ή να προτείνουν κάποιον άλλον πάροχο διαφορετικής υπηρεσίας. Τότε συνεργαζόμενοι με τις κακόβουλες κοινότητες του συστήματος ψεύδονται για τις εμπειρίες τους και προτείνουν κακόβουλες οντότητες για φίλους.

Ένα σύστημα που αντιμετωπίζει όλες τις παραπάνω επιθέσεις δεν μπορεί να αντιμετωπίσει και αυτήν. Για να γίνει αυτό πρέπει να υπάρξει αποσύνδεση της εμπιστοσύνης σε κάποιον ως πάροχο υπηρεσίας, από την εμπιστοσύνη ως πάροχο πληροφοριών. Με αυτόν τον τρόπο θα κρατάμε τους κατασκόπους για να μας παρέχουν σωστές υπηρεσίες αλλά η φήμη της ως παρόχους πληροφοριών θα είναι χαμηλή και άρα δεν θα λαμβάνετε υπόψιν.

\diagramscale{Κακόβουλοι κατάσκοποι}{spies.png}{0.5}

\newpage 

\subsection{Sybil attack}\label{sec:sybil}

Σε αυτήν την επίθεση  Σχ. \ref{fig:sybil.png} κάποιος καταφέρνει να δημιουργήσει ένα τεράστιο αριθμό εικονικών οντοτήτων και να τις χρησιμοποιήσει για τους κακόβουλους σκοπούς τους. Μόλις μειωθεί η φήμη τους τις διαγράφει και κάνει νέες. Για την αντιμετώπισή του πρέπει: \begin{enumerate}

\item Αρχικά να μην υπάρχει πλεονέκτημα σε μία νέα οντότητα σε σχέση με κάποια με χαμηλή φήμη. Αυτό επιτυγχάνετε με την απουσία αρνητικής βαθμολογίας. Έτσι μία νέα οντότητα αντιμετωπίζεται με καχυποψία και θα πρέπει να αποδείξει την αξία της για να ενσωματωθεί

\item Ακόμα πρέπει να υπάρχει ένα ελάχιστο κόστος για την δημιουργία νέων εικονικών οντοτήτων. Στο σύστημα μας προτείνετε να πρέπει ο χρήστης να συνδέσει την οντότητα του με κάποιο user-id. Το οποίο να μην γίνετε αυτόματα (π.χ χρήση Captcha)

\end{enumerate}

\diagramscale{Sybil attack}{sybil.png}{0.3}

\subsection{Ψευδής δήλωση στοιχείων}

Στον Ατζορι έχει προταθεί ο πολλαπλασιασμός τις εμπιστοσύνης σε κάποιον με ένα βάρος για το πόσο εύκολο είναι για αυτόν να συμπεριφερθεί κακόβουλα. Έτσι αναφέρεται ότι θα εμπιστευτεί μία οντότητα πιο εύκολα εάν έχει χαμηλή υπολογιστική ισχύ π.χ. RFID  από μία ισχυρή όπως ένα smartphone. Αλλά είναι δυνατόν κάποιος είτε να ισχυριστεί πώς είναι οντότητα διαφορετικού τύπου από την πραγματικότητα \footnote{ το οποίο ίσως να μπορούσε να ελεγχθεί μέσα από προσπάθεια εντοπισμού των ενεργειών της οντότητας, το οποίο είναι δύσκολο χωρίς κεντρική αρχή που βλέπει όλες τις συναλλαγές}, είτε να έχει τον έλεγχο πραγματικών οντοτήτων τύπου RFID που θα είναι εύκολο να γίνουν έμπιστες και θα τις εκμεταλλεύεται. 

Για αυτό εμείς αντιπροτείνουμαι το βάρος να είναι ανάλογα με το πόσο επικίνδυνο θα ήταν για μία Εικονική Οντότητα να παραπλανηθεί για μία συγκεκριμένη υπηρεσία. Έτσι υπηρεσίες που θα μπορούσε να δώσει ενα RFID  που πιθανώς να ήταν ήσσονος σημασίας θα ζητούνται ακόμα και από Οντότητες μέτριας εμπιστοσύνης, ενώ κρίσιμες υπηρεσίας θα ζητούνται από έμπιστους παρόχους. Το πώς μοντελοποιήθηκε αυτό το βάρος φαίνεται στο ΡΕΦ παρακάτω.
\newpage

\section{RT-IOT}






\newpage
\section{Βασικά Σενάρια Χρήσης}


