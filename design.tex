\chapter{Τύποι επιθέσεων - Αντιμετώπιση με RT-IOT}\label{ch:design}

\section{Επιθέσεις στο Διαδίκτυο των πραγμάτων}

Όπως αναφέρθηκε στο Κεφάλαιο \ref{ch:bkg} ένας από τους κινδύνους παραβίασης της ασφάλειας του Κοινωνικού Διαδικτύου των Πραγμάτων πηγάζει από επιθέσεις κακόβουλων οντοτήτων οι οποίες όμως νομίμως δρουν εντός της κοινωνίας των πραγμάτων. Σε αυτό το κομμάτι του κεφαλαίου θα αναλύσουμε τις βασικότερες επιθέσεις που μπορούν να συμβούν.

\subsection{Μεμονωμένες κακόβουλες Εικονικές Οντότητες}\label{sec:indv}

Οι μεμονωμένες κακόβουλες Εικονικές Οντότητες Σχ.\ref{fig:individual.png} είναι o απλούστερος και ο βασικότερος τύπος επίθεσης. Σε αυτήν την περίπτωση ένα ποσοστό των μελών της κοινότητας έχοντας κακόβουλους σκοπούς προσπαθούν να εκμεταλλευτούν την εμπιστοσύνη των υπόλοιπων και όταν τους ζητηθεί κάποια υπηρεσία ή γνώση αντί να τους εξυπηρετήσουν σωστά στέλνουν δεδομένα που θα οδηγήσουν στο να εκτεθεί ο παραλήπτης σε κίνδυνο ή να οδηγηθεί σε δυσλειτουργία. 

\diagramscale{Μεμονωμένες κακόβουλες Εικονικές Οντότητες}{individual.png}{0.45}
\newpage
Η ύπαρξη κακόβουλων Εικονικών Οντοτήτων αποτελεί την βάση όλων των επιθέσεων και ο εντοπισμός τους είναι ο κύριος στόχος του RT-IOT. Για να βρεθούν χρειάζεται η συλλογική συμφωνία του συστήματος στο ποιες ακριβώς είναι.

Αλλά ακόμα και αν δεν τους εντοπίσει το σύστημα ώστε να σταματήσει να τους προτείνει ως πιθανούς φίλους με καλή φήμη, τα μέλη μπορούν αυτόνομα να προστατευθούν. Έτσι μετά από έναν αριθμό αλληλεπιδράσεων μία Εικονική Οντότητα θα χάσει την όποια εμπιστοσύνη τους στον κακόβουλο δράστη και θα τον απορρίψει από τον κοινωνικό του περίγυρο. Το θέμα είναι πόσο γρήγορα θα το κάνει ώστε να περιορίσει του κινδύνους.



\subsection{Συνεργαζόμενες κακόβουλες Εικονικές Οντότητες}\label{sec:col}

Σε αυτού του είδους της επίθεσης Σχ. \ref{fig:collusion.png} υπάρχει ένας αριθμός κακόβουλων Εικονικών Οντοτήτων όπως περιγράφηκε στην παράγραφο \ref{sec:indv} οι οποίες όμως τώρα έχουν γνώση (μερική ή ολική) για την ύπαρξή και άλλον κακόβουλων Εικονικών Οντοτήτων. Με αυτή λοιπόν την γνώση συνεργάζονται μεταξύ τους και όταν στέλνουν, είτε στην κεντρική αρχή είτε σε άλλα μέλη της κοινότητας, πληροφορίες για να εξαχθεί ο δείκτης φήμης παραποιούν τα δεδομένα τους και να παρουσιάζουν τους κακόβουλους συνεργάτες τους ως φερέγγυους.

\diagramscale{Συνεργαζόμενες κακόβουλες Εικονικές Οντότητες}{collusion.png}{0.4}

Αυτού του είδους η επίθεση είναι πολύ πιο επικίνδυνη από την προηγούμενη. Αυτό συμβαίνει επειδή δεν είναι δυνατόν να βασιστούν οι Εικονικές Οντότητες στην καλή θέληση των μελών της κοινωνίας για να εξαχθεί η φήμη αφού υπάρχει μεγάλη πιθανότητα να ψεύδονται για τις εμπειρίες τους. Για την αντιμετώπιση του πρέπει: \begin{itemize}
\item Στην περίπτωση της κεντρικής αρχής να εντοπιστεί όλη η κοινότητα των κακόβουλων δραστών και να σταματήσει να λαμβάνεται υπόψιν η γνώμη τους κατά την εξαγωγή φήμης.

\item Στην περίπτωση συστημάτων χωρίς κεντρική αρχή πρέπει η κάθε Εικονική Οντότητα να εντοπίσει κάποιου έμπιστους φίλους των οποίων η πιθανότητα να ψεύδονται για τις εμπειρίες τους θα είναι μειωμένη σε σχέση με το σύνολο.
\end{itemize}

Μια εξελιγμένη μορφή αυτής τις επίθεσης είναι η κακόβουλες Εικονικές Οντότητες να προσπαθούν να εντοπίσουν μέσω τις φήμης ανταγωνιστικές προς τις υπηρεσίες που εκμεταλλεύονται Εικονικές Οντότητες και να προσπαθήσουν να χαμηλώσουν την φήμη τους με περαιτέρω ψευδής δηλώσεις για την εμπειρία τους με αυτές.

\subsection{Μερικώς κακόβουλες Εικονικές Οντότητες}\label{sec:part}

Σε αυτού του είδους της επίθεσης Σχ. \ref{fig:partially.png} υπάρχουν κακόβουλες συνεργασίες όπως αναφέρθηκαν στην παράγραφο \ref{sec:col}. Η διαφορά τώρα όμως είναι πώς αυτές οι οντότητες προσπαθούν να νικήσουν το σύστημα φήμης αλλάζοντας την συμπεριφορά τους. Αυτό μπορεί αν συμβεί με 2 τρόπους: 
\begin{enumerate}

\item Αρχικά μπορούν να παρέχουν κακές υπηρεσίες μόνος $ p \ \%  $ των περιπτώσεων ώστε να μην καταποντίζεται η φήμη τους. Δηλαδή μόλις δουν ότι υπάρχει μία κίνηση αποκλεισμού τους, να γίνονται καλές ώστε να παραμείνουν ενσωματωμένες. Σε αυτή την περίπτωση πρέπει το σύστημα που υπολογίζει την φήμη να έχει μνήμη ώστε να βλέπει καχύποπτα τέτοιου είδους συμπεριφορές και να εφαρμόζει αποκλεισμό με το που ξαναγίνουν κακόβουλες. Παρόλα αυτά αυτή η επίθεση δεν είναι τόσο επικίνδυνη επειδή η κακή συμπεριφορά της Εικονικής Οντότητας είναι φραγμένη από το $p$

\item Ένας άλλος τρόπος να ξεγελάσουν το σύστημα είναι να προσθέσουν στο ενεργητικό τους και άλλες υπηρεσίες. Έτσι θα είναι πιο εύκολο να εκμεταλλεύονται μία υπηρεσία για τους κακόβουλους σκοπούς τους αλλά να παραμένουν στο απυρόβλητο από το σύστημα φήμης επειδή παρέχουν άλλες καλά. Για να αντιμετωπιστεί αυτού του είδους η επίθεση πρέπει να υπάρχει διαφοροποίηση των δεικτών φήμης ανά υπηρεσία. Αλλά για να μπορέσει παραμείνει κλιμακώσιμο πρέπει να γίνει ομαδοποίηση σε παρόμοιες υπηρεσίες ώστε να περιοριστεί το εύρος που μπορεί να εκμεταλλευτεί μία κακόβουλη Εικονική Οντότητα.
\end{enumerate}

\diagramscale{Μερικώς κακόβουλες Εικονικές Οντότητες}{partially.png}{0.35}
\newpage
\subsection{Κακόβουλοι κατάσκοποι}\label{sec:spies}

Αυτού του είδους η επίθεση  Σχ. \ref{fig:spies.png} είναι βοηθητική στις \ref{sec:col} και \ref{sec:part}. Οι Εικονικές Οντότητες - κατάσκοποι. Ονομάστηκαν έτσι επειδή στα μάτια του μεγαλύτερου μέρους των μελών του συστήματος παρουσιάζονται ως αξιόπιστες. Αυτό το καταφέρνουν παρέχοντας πάντα μία υπηρεσία καλά. Έτσι αυξάνουν την φήμη τους και την εμπιστοσύνη μεμονωμένων Εικονικών Οντοτήτων προς αυτές. Αυτό όμως που τις κάνει κακόβουλες είναι η συμπεριφορά τους όταν τους ζητηθεί αξιολόγηση ή να προτείνουν κάποιον άλλον πάροχο διαφορετικής υπηρεσίας. Τότε συνεργαζόμενοι με τις κακόβουλες κοινότητες του συστήματος ψεύδονται για τις εμπειρίες τους και προτείνουν κακόβουλες οντότητες για φίλους.

Ένα σύστημα που αντιμετωπίζει όλες τις παραπάνω επιθέσεις δεν μπορεί να αντιμετωπίσει και αυτήν. Για να γίνει αυτό πρέπει να υπάρξει αποσύνδεση της εμπιστοσύνης σε κάποιον ως πάροχο υπηρεσίας, από την εμπιστοσύνη ως πάροχο πληροφοριών. Με αυτόν τον τρόπο θα κρατάμε τους κατασκόπους για να μας παρέχουν σωστές υπηρεσίες αλλά η φήμη της ως παρόχους πληροφοριών θα είναι χαμηλή και άρα δεν θα λαμβάνετε υπόψιν.

\diagramscale{Κακόβουλοι κατάσκοποι}{spies.png}{0.5}

\newpage 

\subsection{Sybil attack}\label{sec:sybil}

Σε αυτήν την επίθεση  Σχ. \ref{fig:sybil.png} κάποιος καταφέρνει να δημιουργήσει ένα τεράστιο αριθμό εικονικών οντοτήτων και να τις χρησιμοποιήσει για τους κακόβουλους σκοπούς τους. Μόλις μειωθεί η φήμη τους τις διαγράφει και κάνει νέες. Για την αντιμετώπισή του πρέπει: \begin{enumerate}

\item Αρχικά να μην υπάρχει πλεονέκτημα σε μία νέα οντότητα σε σχέση με κάποια με χαμηλή φήμη. Αυτό επιτυγχάνετε με την απουσία αρνητικής βαθμολογίας. Έτσι μία νέα οντότητα αντιμετωπίζεται με καχυποψία και θα πρέπει να αποδείξει την αξία της για να ενσωματωθεί

\item Ακόμα πρέπει να υπάρχει ένα ελάχιστο κόστος για την δημιουργία νέων εικονικών οντοτήτων. Στο σύστημα μας προτείνετε να πρέπει ο χρήστης να συνδέσει την οντότητα του με κάποιο user-id. Το οποίο να μην γίνετε αυτόματα (π.χ χρήση Captcha)

\end{enumerate}

\diagramscale{Sybil attack}{sybil.png}{0.3}

\subsection{Ψευδής δήλωση στοιχείων}

Στον Ατζορι έχει προταθεί ο πολλαπλασιασμός τις εμπιστοσύνης σε κάποιον με ένα βάρος για το πόσο εύκολο είναι για αυτόν να συμπεριφερθεί κακόβουλα. Έτσι αναφέρεται ότι θα εμπιστευτεί μία οντότητα πιο εύκολα εάν έχει χαμηλή υπολογιστική ισχύ π.χ. RFID  από μία ισχυρή όπως ένα smartphone. Αλλά είναι δυνατόν κάποιος είτε να ισχυριστεί πώς είναι οντότητα διαφορετικού τύπου από την πραγματικότητα \footnote{ το οποίο ίσως να μπορούσε να ελεγχθεί μέσα από προσπάθεια εντοπισμού των ενεργειών της οντότητας, το οποίο είναι δύσκολο χωρίς κεντρική αρχή που βλέπει όλες τις συναλλαγές}, είτε να έχει τον έλεγχο πραγματικών οντοτήτων τύπου RFID που θα είναι εύκολο να γίνουν έμπιστες και θα τις εκμεταλλεύεται. 

Για αυτό εμείς αντιπροτείνουμαι το βάρος να είναι ανάλογα με το πόσο επικίνδυνο θα ήταν για μία Εικονική Οντότητα να παραπλανηθεί για μία συγκεκριμένη υπηρεσία. Έτσι υπηρεσίες που θα μπορούσε να δώσει ενα RFID  που πιθανώς να ήταν ήσσονος σημασίας θα ζητούνται ακόμα και από Οντότητες μέτριας εμπιστοσύνης, ενώ κρίσιμες υπηρεσίας θα ζητούνται από έμπιστους παρόχους. Το πώς μοντελοποιήθηκε αυτό το βάρος φαίνεται στο ΡΕΦ παρακάτω.
\newpage

\section{RT-IOT}
\subsection{Εισαγωγή}
To RT-IOT (Reputation \& Trust for the Internet Of Things) είναι ένα σύστημα που προτείνουμε για την αντιμετωπίσει των παραπάνω επιθέσεων στο περιβάλλον του Κοινωνικού Διαδικτύου των Πραγμάτων. Ο βασικός στόχος του ήταν να μπορεί να παρέχει μία αποτελεσματική άμυνα προς τις εσωτερικές επιθέσεις, παρέχοντας ταυτόχρονα την δυνατότητα εισόδου και εξόδου Εικονικών Οντοτήτων στο σύστημα 
και λαμβάνοντας υπόψιν των μεγάλο αριθμό Εικονικών Οντοτήτων που θα πρέπει να διαχειριστεί.

Για να δημιουργηθεί ένα τέτοιο σύστημα η βασική σχεδιαστική απόφαση έγκειται στο γεγονός ότι δεν είναι δυνατόν αυτό να γίνει ούτε με μία κεντρική αρχή ούτε κατανεμημένα, εάν η φήμη μίας Εικονικής Οντότητας ήταν ίδια για όλο το σύστημα, όπως συμβαίνει στα περισσότερα συστήματα εμπιστοσύνης/φήμης. Έπρεπε να εγκαταλειφθεί η ιδέα της "αντικειμενικής αλήθειας" για την φήμη του άλλου. Για αυτό κοιτάξαμε την κοινωνία όπου όχι μόνο δεν γνωρίζουμε την φήμη όλων των μελών της αλλά αγνοούμε ακόμα και την ύπαρξη τους.

Με αυτόν τον περιορισμό ως θεμέλιο λίθο χτίσαμε ένα σύστημα με ανοχή σε σφάλματα όπου συνδυάζει ιδέες τόσο από συστήματα με κεντρική αρχή όσο και από συστήματα με κατανεμημένη λογική.

\subsection{Υπολογισμός Εμπιστοσύνης}
Όπως φαίνεται και από στον ορισμό της στην Παρ. \ref{sec:trust}, η εμπιστοσύνη είναι υποκειμενική. Για αυτό και είναι λογικό να ορίζεται και να διαφέρει για κάθε Εικονική Οντότητα.

Για να μοντελοποιήσουμε την εμπιστοσύνη χρειάζεται να μοντελοποιηθεί η εμπειρία. Άρα χρειαζόμαστε μνήμη. Για αυτό κάθε Εικονική Οντότητα έχει log files όπου αποθηκεύει της εμπειρίες των αλληλεπιδράσεών της με άλλες Ε.Ο. Εδικότερα όπως αναφέραμε στην παράγραφο \ref{sec:part} πρέπει να έχουμε διαφορετικά log file για διαφορετικά είδη υπηρεσιών (π.χ. άλλο Log File για υπηρεσίες στο domain "home automation" ,αλλο  στο domain "vehicles" και άλλα για ανταλλαγή γνώσης). Επίσης για γρηγορότερους και ευκολότερους υπολογισμούς δέν έχουμε ένα log file ανα υπηρεσία, αλλά ένα ανά οντότητα εντός της υπηρεσία.

Άρα εάν μία Εικονική Οντότητα ζητάει την υπηρεσία "vehicle", όπου έχει γνωστούς για αυτή τον Β και τον Γ, και επίσης ζητάει και γνώση , όπου έχει γνωστούς για αυτή τον Β και τον Ε. Τα log file είναι όπως στο σχήμα \ref{fig:files.png}
\newpage
\diagramscale{Ιεραρχία Log File}{files.png}{0.4}


Σε κάθε εγγραφή του log file βάζουμε τις εξής στήλες:
\begin{itemize}

\item \textbf{Ικανοποίηση:} Είναι η βαθμολογία που εξάγει μία εικονική οντότητα για την ποιότητα της υπηρεσίας/γνώσης που έλαβε

\item \textbf{Βάρος:} Είναι μία τιμή που καταγράφει η εικονική οντότητα το πόσο κρίσιμη για αυτήν είναι η υπηρεσία που ζήτησε. Αυτή η τιμή χρησιμοποιείτε και για να ορίσει ένα κάτω όριο εμπιστοσύνης κατά την φάση αναζήτησης παρόχου. Επίσης το βάρος μπορεί να επηρεαστεί και από άλλους παράγοντες όπως εάν ανήκουν οι δύο οντότητες στον ίδιο χρήστη(co-ownership),εάν οι χρήστες γνωρίζονται από πρίν(friends), εάν βρίσκονται γεωγραφικά κοντά (co-location), εάν είναι ίδιου τύπου οντότητες (homophily) κ.α

\item \textbf{Εξασθένηση:} Αναγνωρίζοντας την ανάγκη οι νεότερες συναλλαγές να έχουν μεγαλύτερο βάρος από τις παλαιότερες εισάγουμε την έννοια του παράγοντα εξασθένησης (fading effect). Με αυτόν τον τρόπο αναγκάζουμε της Εικονικές οντότητες να έχουν συνεπή συμπεριφορά και να μην μπορούν να παρεκτραπούν εκμεταλλευόμενη την καλή τους ιστορία. Επίσης με τον δείκτη εξασθένισης καταφέρνουμε να κρατήσουμε το σύστημα κλιμακόσημο αφού κάθε Εικονική Οντότητα κρατάει μικρό log file.

\end{itemize}

\begin{table}[H]
    \centering
    \begin{tabular}{ | l | l | l | }
        \hline
        Ικανοποίηση & Βάρος & Εξασθένηση \\ \hline \hline
        0.1 & 0.8 & 1.0  \\ \hline
        1.0 & 0.1 & 0.95  \\ \hline
        0.3 & 0.8 & 0.9  \\ \hline
        0.75 & 0.25 & 0.85  \\ \hline
    \end{tabular}
    \caption{Log File Εικονικής Οντότητας Α για της αλληλεπιδράσεις με την Β}
    \label{tab:log file}
\end{table}

\newpage
Όταν η Εικονική Οντότητα Α θέλει να υπολογίσει την εμπιστοσύνη της στην εικονική οντότητα Β για μία συγκεκριμένη υπηρεσία βρίσκει το κατάλληλο log file και μετά υπολογίζει την μέση τιμή της ικανοποίησης λαμβάνοντας υπόψιν το βάρος και τον παράγοντα εξασθένησης. Άρα 
\begin{equation}
 \mu\  =\ \frac{\sum_{i=1}^{N}\left(s_i\ *\ w_i\ *\ f_i\right)}{W} 
\end{equation}

όπου 
\begin{equation}
 W = \sum_{i=1}^{N}\left(w_i\ *\ f_i\right)
\end{equation}

Το W είναι ο συντελεστής κανονικοποίησής και εξασφαλίζει ότι η μέση τιμή θα παραμείνει εντός του εύρους [0,1]. Η μέση τιμή (μ) είναι μία μέτρηση της ολικής συμπεριφοράς της Εικονικής Οντότητας και μας δείχνει πια είναι η πιθανότερη τιμή ικανοποίησης που θα πάρουμε ένα ζητήσουμε την υπηρεσία από τον συγκεκριμένο πάροχο.

Παρόλα αυτά θέλουμε να ξέρουμε και πόσο σίγουροι μπορούμε να είμαστε για την τιμή του μ. Δηλαδή πόσο μπορεί να αποκλίνει η ικανοποίηση της υπηρεσίας από το μ. Για αυτό υπολογίζουμε και την τυπική απόκλιση της συμπεριφοράς. Για να κάνουμε αρκετά λιγότερες πράξεις το κάνουμε με τον ακόλουθο τύπο ταυτόχρονα με τον υπολογισμός της μέσης τιμής.

\begin{equation}
 \sigma\  =\ \frac{1}{W} \sqrt{\sum_{i=1}^{N}\left(s_i^2\ *\ w_i\ *\ f_i\right)* \ \sum_{i=1}^{N} \left(w_i\ * \ f_i \right)- \left( \sum_{i=1}^{N} s_i\ *\ w_i\ *\ f_i\right)^2}
\end{equation}

Τέλος ορίζουμε την εμπιστοσύνη ως:

\begin{equation}
 T \ = \ \mu \ - \sigma 
\end{equation}

Δηλαδή ,υποθέτοντας πώς η συμπεριφορά της ΕΟ ακολουθεί την κανονική κατανομή \footnote{μπορεί μέσα από στατιστική ανάλυση να βρεθεί η πραγματική κατανομή και να αλλαχθούν τα ποσοστά}, μπορούμε να πούμε ότι εάν ζητήσουμε την υπηρεσία από τον συγκεκριμένο πάροχο έχουμε κάτω από 15\% πιθανότητα να πάρουμε ικανοποίηση χαμηλότερη από Τ και άρα το ρίσκο είναι με το μέρος της Εικονικής Οντότητας.

\begin{table}[H]
    \centering
    \begin{tabular}{ | l | l | l | }
        \hline
        Σύμβολο & Περιγραφή \\ \hline \hline
        $s_i$ & Ικανοποίηση στην συγκεκριμένη συναλλαγή  \\ \hline
        $w_i$ & Βάρος στην συγκεκριμένη συναλλαγή  \\ \hline
        $f_i$ & Εξασθένηση της συγκεκριμένη συναλλαγή  \\ \hline
        μ 	& Μέση τιμή \\ \hline
        σ   & Τυπική απόκλιση \\ \hline
        Τ   & Εμπιστοσύνη \\ \hline
        R   & φήμη \\ \hline
    \end{tabular}
    \caption{Πίνακας συμβόλων}
    \label{tab:symbols}
\end{table}
\newpage
\subsection{Υπολογισμός Φήμης}
\section{Βασικά Σενάρια Χρήσης}


