\chapter{Εισαγωγή}\label{ch:intro}

Όταν το 1969 γεννήθηκε το ARPANET η ιδέα που το ώθησε  ήταν η κοινή χρήση υπολογιστικών πόρων από απομακρυσμένες περιοχές. Δηλαδή εάν σε κάποιο ερευνητικό κέντρο χρειαζόταν να γίνουν πολλοί υπολογισμοί, να μπορούσαν να γίνουν σε κάποιο εξειδικευμένο υπολογιστικό σύστημα που ήταν γεωγραφικά απομακρυσμένο από το ερευνητικό κέντρο.

H ιδέα αυτή ποτέ δεν δούλεψε πραγματικά - για αρχή, όλοι οι υπολογιστές είχαν διαφορετικά λειτουργικά συστήματα και προγράμματα, ενώ η χρήση του μηχανήματος κάποιου άλλου ήταν πολύ δύσκολή. Επίσης μέχρι να υλοποιηθεί το ARPANET η τεχνολογία είχε φτάσει στο σημείο να μην το έχει ανάγκη αφού είχαν εμφανιστεί οι πρώτοι προσωπικοί υπολογιστές και ο καταμερισμός χρόνου σε υπολογιστικά συστήματα δεν πρόσφερε ποία τόσα οφέλη.

Έτσι είναι λογικό να πούμε ότι το ARPANET απέτυχε στο σκοπό του, αλλά στη διαδικασία αυτή έκανε μερικές σημαντικές ανακαλύψεις που είχαν ως αποτέλεσμα τη δημιουργία των τεχνολογιών του πρώτου Διαδικτύου. Σε αυτές περιλαμβάνονταν το e-mail, η μεταγωγή πακέτων εφαρμογών, και φυσικά η ανάπτυξη του Transport Control Protocol - Internet Protocol - Internet Protocol ή TCP / IP.

Στην συνέχεια και λόγω της ραγδαίας εξάπλωσης των προσωπικών υπολογιστών το Internet έγινε ή ποίο μεγάλη απόδειξη της ισχύς του νόμου του Metclafe κατά τον οποίο:%\cite{Metclafe}
\begin{quotation}
"Η αξία ενός δικτύου είναι ανάλογη με το τετράγωνο του αριθμού των κόμβων, δηλαδή του αριθμού των χρηστών του δικτύου."
\end{quotation}

Σήμερα, το Διαδίκτυο με πάνω από 3 δισεκατομμύρια χρήστες είναι το βασικότερο μέσο 
εύρεσης και ανταλλαγής δεδομένων της ανθρωπότητας Ένας πολίτης του 21ου αιώνα δεν χρειάζεται να γνωρίζει κάθε μικρή πληροφορία, το μόνο που χρειάζεται να ξέρει είναι σε ποίο σημείο του διαδικτύου μπορεί να την βρει. 

Έτσι τώρα, βρισκόμαστε σε μία μεταβατική περίοδο. Το Διαδίκτυο είναι ευρέως αποδεκτό και αναγκαίο, αλλά η έμφυτη ανθρώπινη περιέργεια καθώς και η δημιουργικότητα μας οδηγούν στο επόμενο βήμα. Στο Διαδίκτυο των Πραγμάτων.

Ποία είναι όμως η διάφορα του Διαδικτύου από το Διαδίκτυο των Πραγμάτων; 
\newpage
Τα λεγόμενα του Kevin Ashton, συνιδρυτή και εκτελεστικού διευθυντή του Auto-ID Center στο ΜΙΤ είναι ξεκάθαρα:%\cite{Ashton}
\begin{quotation}

"Σήμερα οι υπολογιστές - και, ως εκ τούτου, το Διαδίκτυο - εξαρτώνται σχεδόν εξ ολοκλήρου από τα ανθρώπινα όντα για πληροφορίες. Σχεδόν το σύνολο των περίπου 50 petabytes δεδομένων διαθέσιμων στο Διαδίκτυο, συλλαμβάνεται και δημιουργείται από τον άνθρωπο με την πληκτρολόγηση, το πάτημα ενός κουμπιού εγγραφής, την λήψη μιας ψηφιακής φωτογραφίας ή την σάρωση ενός barcode.


Το πρόβλημα είναι πώς, οι άνθρωποι έχουν περιορισμένο χρόνο, προσοχή και ακρίβεια. Γιαυτό και δεν είναι πολύ καλοί στο καταγραφή των δεδομένων που αφορούν τα πράγματα του γύρω κόσμου. Αν είχαμε υπολογιστές που ήξεραν οτιδήποτε μπορούσαν να ξέρουν για τα πράγματα γύρω τους - με τη χρήση δεδομένων που συνέλεξαν αυτόνομα χωρίς καμία βοήθεια από εμάς - τότε θα ήταν σε θέση να παρακολουθούν και να μετράνε τα πάντα. Έτσι θα μειωνόντουσαν σε μεγάλο βαθμό η σπατάλη αγαθών, οι απώλειες από λάθη και το γενικότερο κόστος. Θα γνωρίζαμε από το πότε τα πράγματα θα χρειάζονται αντικατάσταση ή επισκευή  έως και το αν ήταν φρέσκα ή όχι." 
\end{quotation}


\section{Κοινωνικό Διαδίκτυο των Πραγμάτων και Προβλήματα Εμπιστοσύνης}

Η Διπλωματική αυτή εργασία γεννήθηκε για να αντιμετωπισθούν ανησυχίες σχετικά με τις δράσεις των Εικονικών Οντοτήτων(Virtual Entities), οι οποίες αποτελούν την αναπαράσταση των πραγμάτων στον ψηφιακό κόσμο. Όταν λοιπόν σε αυτές τις Εικονικές Οντότητες προσθέσουμε την δυνατότητα να έχουν κοινωνικούς δεσμούς μέσο φιλίας, μέσο οικογένειας(ανήκουν στον ίδιο χρήστη) ή και μέσο ομοιότητας (έχουν τον ίδιο ρόλο στο σύστημα), τότε βρισκόμαστε στο Κοινωνικό Διαδίκτυο των Πραγμάτων \footnote{www.social-iot.org/}. Ειδικότερα αυτή η εργασία έγινε στα πλαίσια του COSMOS(Cultivate resilient smart Objects for Sustainable city applicatiOn) \footnote{www.iot-cosmos.eu/}, σκοπός του οποίου είναι η δημιουργία έξυπνων αντικειμένων, προκειμένου να καταστεί δυνατή μια έξυπνη πόλη. Στο COSMOS

\begin{itemize}
\item
Τα πράγματα θα είναι σε θέση να μάθουν βασιζόμενα σε εμπειρίες άλλον,
\item
 ενώ μέσα από την απόκτηση και την ανάλυση της γνώσης τα πράγματα θα γνωρίζουν τις συνθήκες και τα γεγονότα που συμβαίνουν και ανάλογα θα μεταβάλουν τη συμπεριφορά τους.
\item Οι διαχειριστικές αποφάσεις θα λαμβάνονται σε πραγματικό χρόνο για κάθε Εικονική Οντότητα βασιζόμενες στην ασφάλεια των πραγμάτων, την γεωγραφική τους θέση, τις σχέσεις που έχουν με άλλες εικονικές οντότητες καθώς και άλλες ψηφιακές πληροφορίες που θα εξάγονται από
\item Complex Event Processing και άλλες τεχνολογίες Κοινωνικών Mέσων ώστε να εντοπίζεται η χρήσιμη πληροφορία μεσα στον τεράστιο αριθμό δεδομένων (Big Data)
\end{itemize}
\newpage

Στα πλαίσια λοιπόν του COSMOS  υπήρξε η ανάγκη να υπάρχει γνώση για το πόσο έμπιστος μπορεί να θεωρηθεί ένα πάροχος πληροφοριών ή/και υπηρεσιών. Με αυτή τη γνώση θα ήταν δυνατό: 
\begin{itemize}
	 \item Να εντοπίζεται γρήγορα η πιο αξιόπιστη πηγή πληροφοριών και να μειώνεται η επικοινωνία με ταυτόχρονη βελτιστοποίηση της απόδοσης
	 
 	 \item Να μπορεί μία Εικονική Οντότητα να κρίνει αυτόνομα πόσο πιθανό είναι να ικανοποιηθεί από μία προσφερόμενη υπηρεσία πριν προβεί στην δοσοληψία επειδή
 	 \begin{itemize}
 	 	\item μπορεί κάποια υπηρεσία να είναι προϊόν
 	 	\item μπορεί να μας δοθεί γνώση που θα ρισκάρει την ασφάλεια του συστήματος ή/και των τελικών χρηστών
 	 	\end{itemize}
\end{itemize}
 

\section{Δομή της Διπλωματική Εργασίας} 
Η διπλωματική εργασία δομείται ως εξής:

\begin{description} \item[Chapter \ref{ch:bkg}] \hfill \\
Σε αυτό το κεφάλαιο γίνεται μία εισαγωγή στη βασικές ιδέες που χαρακτηρίζουν γενικά τα συστήματα εμπιστοσύνης - φήμης. Στην συνέχεια αναλύονται οι βασικές αρχιτεκτονικές τους καθώς και οι διάφοροι τρόποι εξαγωγής των δεικτών φήμης και εμπιστοσύνης. Τέλος παρουσιάζονται βασικά συστήματα φήμης και εμπιστοσύνης που χρησιμοποιούνται σε διάφορους τομείς της υπολογιστικής επιστήμης.

\item[Chapter \ref{ch:design}] \hfill \\
Σε αυτό το κεφάλαιο παρουσιάζονται τα είδη των απειλών που μπορούν να προκύψουν και προτάσεις αντιμετώπισής τους. Παράλληλα περιγράφεται ο σχεδιασμός του συστήματος και δίνονται βασικά παραδείγματα εκτέλεσης.

\item[Chapter \ref{ch:implementation}] \hfill \\
Σε αυτό το κεφάλαιο παρουσιάζεται το εργαλείο προσημειώσεων TRMSim-WSN.
Πάνω σε αυτό προστέθηκε ένα μοντέλο του συστήματος φήμης εμπιστοσύνης και αναλύεται ο τρόπος που έγινε αυτό.

\item[Chapter \ref{ch:evaluation}] \hfill \\
EVALUATION
We cite our experience of using BlkKin in Archipelago and RADOS instrumentation
and its use as a debugging and an alerting mechanism.

\item[Chapter \ref{ch:conclusion}] \hfill \\
CONCULSION-FUTURE WORK
We provide some concluding remarks and give some future work that could be done
to improve and evolve BlkKin.
\end{description}
